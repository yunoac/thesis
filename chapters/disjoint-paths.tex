\chapter{Disjoint sr-paths}

Find disjoint paths in a graph is a very well estabilished problem.

In networking, it has several applications such has bla bla

In this chapter we study this problem of finding disjoint paths in a segment routing
setting. 

\section{Finding disjoint paths in a graph}

In this sections we describe several problems related to finding disjoint paths in a graph.
We will see that small changes in the problem definition can have drastic effect on its 
complexity class.

It its most basic form, we have two nodes on the graph $s$ and $t$ that we wish to connect
and we seek to find a maximum cardinality set of edge disjoint paths betweem them.

\begin{problem}{Maximum set of edge disjoint paths}
\label{problem:maxedgedisjoint}
\textbf{Input:} A graph $G = (V, E)$ and two distinct nodes $s, t \in V$.

\textbf{Output:} A set of edge disjoint $s$-$t$ paths $\{ p_1, \ldots, p_k\}$ of maximum cardinality.
\end{problem}

Problem \ref{problem:maxedgedisjoint} is solvable in polynomial time.


\section{Disjoint sr-paths}



\begin{definition}
Two sr-path $p_1$ and $p_2$ are disjoint if and only if $\forw(p_1) \cap \forw(p_2) = \emptyset$.
\end{definition}

