\begin{definition}
Let $G$ be a network. A sr-path $\sr{p}$ on $G$ is a sequence $\langle x_1, \ldots, x_l \rangle$ such that each $x_i \in V(G) \cup E(G)$.
\end{definition}

\begin{definition}
Let $G$ be a network and $\sr{p} = \langle x_1, \ldots, x_l \rangle$. If $x_i \in V(G)$ we define $x^1_i = x^2_i = x_i$ and
if $x_i = (u_1, u_2) \in E(G)$ we define $x^1_i = u_1$ and $x^2_i = u_2$. 
\end{definition}

\begin{definition}
Let $G$ be a network and $\sr{p} = \langle x_1, \ldots, x_l \rangle$. We say that $\sr{p}$ is a path from node $x^1_1$ to
$x^2_l$. We say that $x^1_1$ is the first node of $\sr{p}$ and $x^2_l$ is the last node of $\sr{p}$. A sr-path that
starts and ends at the same node is called a sr-cycle.
\end{definition}

\begin{definition}
Let $G$ be a network and $\sr{p} = \langle x_1, \ldots x_l \rangle$ a sr-path on $G$. We define the cost of $x_i$ as
\[ \cost(x_i) =
  \begin{cases}
    1 & \quad \text{if } x_{i} \in V(G) \\
    2 & \quad \text{otherwise}
  \end{cases}
\]
We define the segment cost of $\sr{p}$ as
$$
\cost(\sr{p}) = \sum_{i = 1}^l \cost(x_i)
$$
\end{definition}

\begin{definition}
Let $G$ be a network and $\sr{p} = \langle x_1, \ldots, x_l \rangle, \sr{q} = \langle y_1, \ldots, y_r \rangle$ be two sr-paths on $G$. We 
define the concatenation of $\sr{p}$ and $\sr{q}$ as
$$
\sr{p} + \sr{q} = \langle x_1, \ldots, x_l, y_1, \ldots, y_r \rangle
$$
\end{definition}

\begin{lemma}
\label{lemma:srsumcost}
Let $G$ be a network and $\sr{p} = \langle x_1, \ldots, x_l \rangle, \sr{q} = \langle y_1, \ldots, y_r \rangle$ be two sr-paths on $G$. Then
$$
\cost(\sr{p} + \sr{q}) = \cost(\sr{p}) + \cost(\sr{q}) 
$$
\end{lemma}

\begin{proof}
\begin{align*}
\cost(\sr{p} + \sr{q}) & = \cost(\langle x_1, \ldots, x_l, y_1, \ldots, y_r) \\
& = \sum_{i = 1}^l \cost(x_i) + \sum_{i = 1}^r \cost(y_i) \\
& = \cost(\sr{p}) + \cost(\sr{q})
\end{align*}
\end{proof}

\begin{definition}
Let $G$ be a network and $\sr{p} = \langle x_1, \ldots, x_l \rangle$ be a sr-path from
$x$ to $y$ and $\sr{q} = \langle y_1, \ldots, y_r \rangle$ be a sr-path from $y$ to $z$, where $x, y, z \in V(G)$.
We  define the concatenation of $\sr{p}$ and $\sr{q}$ as
\[ \sr{p} \oplus \sr{q} =
  \begin{cases}
    \langle x_1, \ldots, x_l, y_2, \ldots, y_r \rangle & \quad \text{if } x_l = y_1 = y \in V(G) \\
    \langle x_1, \ldots, x_l, y_2, \ldots, y_r \rangle & \quad \text{if } x_l \in E(G) \text{ and } y_1 \in V(G) \\
    \langle x_1, \ldots, x_{l - 1}, y_1, \ldots, y_r \rangle & \quad \text{if } x_l \in V(G) \text{ and } y_1 \in E(G) \\
    \langle x_1, \ldots, x_l, y_1, \ldots, y_r \rangle & \quad \text{if } x_l \in E(G) \text{ and } y_1 \in E(G) \\
  \end{cases}
\]
\end{definition}

\begin{lemma}
\label{lemma:concat-cost}
Let $G$ be a network and $\sr{p} = \langle x_1, \ldots, x_l \rangle$ be a sr-path from
$x$ to $y$ and $\sr{q} = \langle y_1, \ldots, y_r \rangle$ be a sr-path from $y$ to $z$, where $x, y, z \in V(G)$.
Then
\[ \cost(\sr{p} \oplus \sr{q}) =
  \begin{cases}
    \cost(\sr{p}) + \cost(\sr{q}) & \quad \text{if } x_l \in E(G) \text{ and } y_1 \in E(G) \\
    \cost(\sr{p}) + \cost(\sr{q}) - 1 & \quad \text{otherwise} \\
  \end{cases}
\]
\end{lemma}

\begin{proof}
If $x_l \in E(G)$ and $y_1 \in E(G)$ then $\sr{p} \oplus \sr{q} = \sr{p} + \sr{q}$ so the result comes from 
Lemma \ref{lemma:srsumcost}. Otherwise, in each case $\sr{p} \oplus \sr{q}$ has the same elements as $\sr{p} + \sr{q}$ exept
that we removed an node segments with cost $1$. Thus, by Lemma \ref{lemma:srsumcost}, we have that
$$
\cost(\sr{p} \oplus \sr{q}) = \cost(\sr{p} + \sr{q}) - 1 = \cost(\sr{p}) + \cost(\sr{q}) - 1
$$
\end{proof}

\begin{definition}
Let $G$ be a network and $\sr{p} = \langle x_1, \ldots, x_l \rangle$ a sr-path on $G$. We define the set of nodes in $\sr{p}$ as
$$
V(\sr{p}) = \bigcup_{i = 1}^l \{x^1_i, x^2_i \} \quad \cup \quad \bigcup_{i = 2}^l V(\sp(x^2_{i - 1}, x^1_i))
$$
We define the set of edges of $\sr{p}$ as
$$
E(\sr{p}) = \left\{ x_i \mid \textrm{$x_i$ is an adjacency segment} \right\} \quad \cup \quad \bigcup_{i = 2}^l E(\sp(x^2_{i - 1}, x^1_i))
$$
For simplicity, we often write $v \in \sr{p}$ to mean that $v \in V(\sr{p})$ and $e \in \sr{p}$ when $e \in E(\sr{p})$.
\end{definition}

\begin{definition}
A sr-path $\sr{p} = \langle x_1, \ldots, x_l \rangle$ is said to be deterministic if for each $i \in \{2, \ldots, l\}$
there exists a unique IGP shortest path between $x^2_{i - 1}$ and $x^1_i$.
\end{definition}

\begin{lemma}
\label{lemma:deterministic-concat}
Let $G$ be a netwrok and $\sr{p}$ be a deterministic sr-path from
$x$ to $y$ and $\sr{q}$ be a deterministic sr-path from $y$ to $z$ 
then $\sr{p} \oplus \sr{q}$ is a deterministic sr-path
from $x$ to $z$.
\end{lemma}

\begin{proof}
Write $\sr{p} = \langle x_1, \ldots, x_l \rangle$ and 
$\sr{q} = \langle y_1, \ldots, y_r \rangle$.
Let $z_i, z_{i + 1}$ be consecutive elements of $\sr{p} \oplus \sr{q}$. 
We need to prove that
there exists a unique shortest path between $z^2_i$ and $z^1_{i + 1}$.
If both $z_i, z_{i + 1}$ belong to $\sr{p}$ or $\sr{q}$ then this is true since both
$\sr{p}$ and $\sr{q}$ are deterministic. Otherwise, there are four cases that we need
to consider.

\emph{Case 1:} $x_l = y_1$ are both node segments. In this case we know that
$\sr{p} \oplus \sr{q} = \langle x_1, \ldots, x_l, y_2, \ldots, y_r \rangle$.
Thus, $z_i = x_l = y_1$ and $z_{i + 1} = y_2$. Hence, since $\sr{q}$ is deterministic,
there exists a unique shortest path between $z^2_i = y^2_1$ and $z^1_{i + 1} = y^1_2$.

\emph{Case 2:} $x_l,  y_1$ are both adjacency segments and 
$x^2_l = y^1_1$. In this case we know that
$\sr{p} \oplus \sr{q} = \langle x_1, \ldots, x_l, y_1, y_2, \ldots, y_r \rangle$.
Hence, $z_i = x_l$ and $z_{i + 1} = y_1$ so $z^2_i = x^2_l = y^1_1 = z^1_{i + 1}$ so
the unique shortst path between $z^2_i$ and $z^1_{i + 1}$ is the empty path.

\emph{Case 3:} $x_l$ is an adjacency segment and $y_1$ is a node segment such that
$x^2_l = y_1$. By definition, 
$\sr{p} \oplus \sr{q} = \langle x_1, \ldots, x_l, y_2, \ldots, y_r \rangle$.
Thus, $z_i = x_l$ and $z_{i + 1} = y_2$. Thus, since $\sr{q}$ is deterministic,
there exists a unique shortest path between $z^2_i = x^2_l = y^2_1$ and $z^1_{i + 1} = y^1_2$.

\emph{Case 4:} $x_l$ is a node segment and $y_1$ is an adjacent segment such that
$y^1_l = x_l$. Therefore $\sr{p} \oplus \sr{q} = \langle x_1, \ldots, x_{l - 1}, y_1, y_2, \ldots, y_r \rangle$.
Thus, $z_i = x_{l - 1}$ and $z_{i + 1} = y_1$. Thus, since $\sr{p}$ is deterministic,
there exists a unique shortest path between $z^2_i = x^2_{l - 1}$ and $z^1_{i + 1} = y^1_2 = x^1_l$.
\end{proof}



%\begin{definition}
%Let $G$ be a network, $s \in V(G)$ and $k \geq 0$ an integer. We define the deterministic node reachability of $v$ with cost
%at most $k$ as
%$$
%\nreach(k, s) = \left\{ v \in V(\sr{p}) \mid \textrm{$\sr{p}$ is a deterministic sr-path starting at $s$ with $\cost(\sr{p}) \leq k$} \right\}
%$$
%\end{definition}

---------------------------------------------------------------------------------------------------

\begin{definition}
Let $G$ be a network and $s \in V(G)$. For any integer $k \geq 1$ we define
\begin{align*}
\nreach^n(k, s) = & \ \textrm{set of nodes $v \in V(G)$ such that there exists a deterministic} \\ 
                & \ \textrm{sr-path $\sr{p} = \langle x_1, \ldots, x_l, v \rangle$ of segment cost at most $k$},
\end{align*}
\begin{align*}
\nreach^e(k, s) = & \ \textrm{set of nodes $v \in V(G)$ such that there exists a deterministic} \\ 
                  & \ \textrm{sr-path $\sr{p} = \langle x_1, \ldots, x_l, (u, v) \rangle$ of segment cost at most $k$}
\end{align*}
and
$$
\nreach(k, s) = \nreach^n(k, s) \cup \nreach^e(k, s)
$$
\end{definition}

\begin{definition}
Let $G$ be a network and $s \in V(G)$. For any integer $k \geq 1$ we define
\begin{align*}
\nreach(k, s) = & \ \textrm{set of nodes $v \in V(G)$ such that there exists a deterministic} \\ 
                & \ \textrm{sr-path $\sr{p}$ from $s$ to $v$ of segment cost at most $k$}
\end{align*}
and
\begin{align*}
\ereach(k, s) = & \ \textrm{set of edges $e \in E(G)$ such that there exists a deterministic} \\ 
                & \ \textrm{sr-path of segment cost at most $k$, $\sr{p} = \langle x_1, \ldots, x_l \rangle$, from} \\
                & \ \textrm{$s$ to $u_2$ such that either $e = x_l$ or $x_l$ is a node segment and} \\
                & \ \textrm{$e$ is the last edge of the unique shortest path between $x^2_{l - 1}$} \\
                & \ \textrm{and $x_l$}
\end{align*}
\end{definition}

\begin{definition}
Let $G$ be a netwrok and $v \in V(G)$. We define the shortest path reach of $v$ as
$$
\spreach(v) = \left\{ u \in V(G) \mid \textrm{there is a unique shortest path between $v$ and $u$} \right\}
$$
\end{definition}


\begin{theorem}
\label{thm:nreach}
Let $G$ be a network, $s \in V(G)$ and an integer $k \geq 3$. It holds that
\begin{align*}
\nreach(1, s) = & \ \{ s \} \\
\nreach(2, s) = & \ \{ v \in V(G) \mid  \sp(s, v) \textrm{ is a path} \} \cup \{ u \mid (s, u) \in E(G) \} \\
\nreach(k, s) = & \ \bigcup_{v \in \nreach(k - 1, s)} \spreach(v) \quad \cup \\ 
& \ \bigcup_{v \in \nreach(k - 2, s)}  \left\{ u_2 \mid (u_1, u_2) \in E(G) \wedge u_1 \in \spreach(v) \right\}
\end{align*}
\end{theorem}

\begin{proof}
For $k = 1$ the only sr-path of cost $1$ starting from $s$ is $\langle s \rangle$. Thus $\nreach(1, s) = \{ s \}$.
For $k = 2$, a sr-path of cost $2$ starting at $s$ has either the form $\langle s, v \rangle$ where
$v \in V(G)$ or $\langle (s, v) \rangle$ where $(s, v) \in E(G)$. In the first case, since we need the sr-path to
be deterministic, only $v \in V(G)$ such that $\sp(s, v)$ is a path yield a deterministic sr-path. In the second case,
a path of the form $\langle (s, v) \rangle$ is always deterministic so each such edge yields a valid path.

Let $k \geq 3$. 

$(\subseteq)$ Suppose that $u \in \nreach(k, s)$. Then there exists a deterministic sr-path 
$\sr{p} = \langle x_1, \ldots, x_l \rangle$ from $s$ to $u$ of segment cost at most $k$.
%Since $\sr{p}$ is deterministic, so are $\langle x_1, \ldots, x_{l - 1} \rangle$ and
%$\langle x^2_{l - 1}, x_l \rangle$.

\emph{Case 1:} $x_l = u$ is a node segment.
Then $\langle x_1, \ldots, x_{l - 1} \rangle$ is a deterministic
sr-path from $s$ to $x^2_{l - 1}$ with segment cost at most $k - 1$. By determinism of $\sr{p}$, there is a unique shortest path
between $x^2_{l - 1}$ and $x^1_l = x_l$. Therefore, $u \in \spreach(s)$. Thus letting $v = x^2_{l - 1}$, we have that
$v \in \nreach(k - 1, s)$ so
$$
u \in \bigcup_{v \in \nreach(k - 1, s)} \spreach(v).
$$

\emph{Case 2:} $x_l$ is an adjacency segment.
Write $x_l = (u_1, u_2)$. Then $\langle x_1, \ldots, x_{l - 1} \rangle$ is a deterministic sr-path 
from $s$ to $x^2_{l - 1}$ with segment cost at most $k - 2$ and by determinism, 
$u_1 = x^1_l \in \spreach(x^2_{l - 1})$. Thus letting $v = x^2_{l - 1}$,
we have that $v \in \nreach(k - 2, s)$ and $u_1 \in \spreach(v)$ proving that
$$
u = u_2 \in \bigcup_{v \in \nreach(k - 2, s)}  \left\{ u_2 \mid (u_1, u_2) \in E(G) \wedge u_1 \in \spreach(v) \right\}
$$

$(\supseteq)$
Let $u \in \bigcup_{v \in \nreach(k - 1, s)} \spreach(v)$. Then there exists $v \in \nreach(k - 1, s)$
such that $u \in \spreach(v)$. Let $\sr{p} = \langle x_1, \ldots, x_l \rangle$ be a deterministic sr-path from $s$ to $v$ with
$\cost(\sr{p}) \leq k - 1$. Since $u \in \spreach(v)$, there is a unique shortest path from $v$ to $u$ so the
sr-path $\langle x_1, \ldots, x_l, u \rangle$ is a deterministic sr-path from $s$ to $u$ with segment cost
at most $k - 1 + 1 = k$. Thus $u \in \nreach(k, s)$.

Let $u \in \bigcup_{v \in \nreach(k - 2, s)}  \left\{ u_2 \mid (u_1, u_2) \in E(G) \wedge u_1 \in \spreach(v) \right\}$. 
Then there exists $v \in \nreach(k - 2, s)$ and $(u_1, u_2) \in E$ such that $u_1 \in \spreach(s)$ and
$u_2 = u$. Let $\sr{p} = \langle x_1, \ldots, x_l \rangle$ be a sr-path from $s$ to $v$ with segment
cost at most $k - 2$. Then $\langle x_1, \ldots, x_l, (u_1, u_2) \rangle$ is a deterministic sr-path from $s$ to $u$ with
segment cost at most $k - 2 + 2 = k$ from $s$ to $u$ so $u \in \nreach(k, v)$.
\end{proof}

\begin{theorem}
Let $G$ be a network $s \in V(G)$, $e = (u_1, u_2) \in E(G)$ and $k \geq 1$ an integer. There exsits a deterministic sr-cycle $\sr{c}$
from $s$ to $s$ of cost at most $k$ such that $e \in \sr{c}$ if and only if there exist integers $k_1, k_2 \geq 1$, $x \in \nreach(k_1, s)$, $y$ such that $s \in \nreach(k_2, y)$ 
and one of the two following conditions holds
\begin{enumerate}[(1)]
 \item $k_1 + k_2 = k$, $y \in \spreach(x)$ and $e \in \sp(x, y)$
 \item $k_1 + k_2 = k - 2$, $u_1 \in \spreach(x)$ and $y \in \spreach(u_2)$
\end{enumerate}
\end{theorem}

\begin{proof}
Let $G$ be a network $s \in V(G)$, $e = (u_1, u_2) \in E(G)$ and $k \geq 1$ an integer.

$(\Rightarrow)$ Assume that there exists a deterministic sr-cycle $\sr{c}$ from $s$ to $s$
with $\cost(\sr{c}) \leq k$ such that $e \in \sr{c}$. Write $\sr{c} = \langle x_1, \ldots, x_l \rangle$.
Since $e \in \sr{c}$ there exists $i \in \{1, \ldots, l\}$ such that either $x_i = e$ or $i < l$ and
$e \in \sp(x^2_{i}, x^1_{i + 1})$.

\emph{Case 1:} $e \in \sp(x^2_{i}, x^1_{i + 1})$. Let $x = x^2_{i}$ and $y = x^1_{i + 1}$. Then $\sr{p} = \langle x_1, \ldots, x_{i} \rangle$ is a deterministic sr-path
from $s$ to $x$ of cost, say, $k_1$ and $\sr{q} = \langle x_{i + 1}, \ldots, x_l \rangle$ is a deterministic sr-path from $y$ to $s$ of cost $k_2 \leq k - k_1$. 
Therefore, $x \in \nreach(k_1, s)$, $s \in \nreach(k_2, y)$. Since $\sr{c}$ is deterministic, there is a unique shortest path
from $x$ to $y$ so $y \in \spreach(x)$. Since by hyphothese $e \in \sp(x, y)$, condition $(1)$ holds.

\emph{Case 2:} $e = x_i$. Let $x = x^2_{i - 1}$ and $y = x^1_{i + 1}$. Then $\sr{p} = \langle x_1, \ldots, x_{i - 1} \rangle$ is a deterministic sr-path
from $s$ to $x$ of cost, say $k_1$, and $\sr{q} = \langle x_{i + 1}, \ldots, x_l \rangle$ from $y$ to $s$ of cost, say $k_2$, such that
$k_1 + k_2 = \cost(\sr{p}) + \cost(\sr{q}) = \cost(\sr{p}) - 2 \leq k - 2$. Thus $x \in \nreach(k_1, s)$ and $s \in \nreach(k_2, y)$.
Since $\sr{c}$ is deterministic, there is a unique shortest path from $x = x^2_{i - 1}$ and $u_1 = x^1_i$. For the same reason,
there exists a unique shortest path from $u_2 = x^2_i$ to $y = x^1_{i + 1}$.  Thus $u_1 \in \nreach(2, x)$ and $y \in \spreach(u_2)$ so that
condition $(2)$ holds.

Note that in the first case we have $k_1 + k_2 \leq k$ and in the second $k_1 + k_2 \leq k - 2$ instead of the equalities. 
This is not a problem because if we have a solution with $k_1' + k_2' < k$ we also have a solution with longer paths. Same same is
true for $k_1' + k_2' < k - 2$. One way to do so is to add the source $s$ enough times so that both paths have the desired segment cost.

$(\Leftarrow)$ Assume that there exist integers $k_1, k_2 \geq 1$, $x \in \nreach(k_1, s)$, $y$ such that $s \in \nreach(k_2, y)$ 
and either $(1)$ or $(2)$ holds. Since $x \in \nreach(k_1, s)$ there exists a deterministic sr-path
$\sr{p} = \langle x_1, \ldots, x_l \rangle$ from $s$ to $x$ with $\cost(\sr{p}) \leq k_1$. In the same way, since $s \in \nreach(k_2, y)$,
there exists a deterministic sr-path $\sr{q} = \langle y_1, \ldots, y_r \rangle$ from $y$ to $s$ with $\cost(\sr{q}) \leq k_2$.

\emph{Case 1:} $k_1 + k_2 = k$, $y \in \spreach(x)$ and $e \in \sp(x, y)$. 
Let $\sr{c} = \sr{p} + \sr{q} = \langle x_1, \ldots, x_l, y_1, \ldots, y_r \rangle$. 
This sr-path is deterministic because since $y^1_1 = y \in \spreach(x) = \spreach(x^2_l)$. 
For the other indexes, the unicity of shortest paths comes from the fact that both $\sr{p}$ and $\sr{q}$
are deterministic. Since $\sr{c}$ goes from $s$ to $s$ and has cost $k_1 + k_2 = k$, $\sr{c}$ is a sr-cycle of cost at most $k$ from $s$ to $s$.
It contains $e$ because $e \in \sp(x, y) = \sp(x^2_l, y^1_1)$.

\emph{Case 2:} $k_1 + k_2 = k - 2$, $u_1 \in \spreach(x)$ and $y \in \spreach(u_2)$. Let $\sr{c} = \langle x_1, \ldots, x_l, e, y_1, \ldots, y_r \rangle$.
Since $u_1 \in \spreach(x)$ and $y \in \spreach(u_2)$ we have that $\sr{c}$ is deterministic. As before, for the other indexes determinism comes from the 
determinism of $\sr{p}$ and $\sr{q}$. Finally, $\cost(\sr{c}) = \cost(\sr{p}) + \cost(\sr{q}) + \cost(e) = k_1 + k_2 + 2 \leq k - 2 + 2 = k$. Clearly $e \in \sr{c}$
so the proof is complete.
\end{proof}


------------------------------------------------------------

\todo{this needs to be fixed, it still uses $\ereach(2, v)$ assuming that it mean unique shortest path}

\begin{theorem}
\label{thm:ereach}
Let $G$ be a network, $s \in V(G)$ and an integer $k \geq 3$. It holds that
\begin{align*}
\ereach(1, s) = & \ \emptyset \\
\ereach(2, s) = & \ \{ e \in E(G) \mid \sp(s, v) \textrm{ is a path with last edge $e$} \} \quad \cup \\
                & \ \{ (s, u) \mid (s, u) \in E(G) \} \\
\ereach(k, s) = & \ \bigcup_{v \in \nreach(k - 1, s)} \ereach(2, v) \quad \cup \\ 
& \ \bigcup_{v \in \nreach(k - 2, s)}  \left\{ (u_1, u_2) \mid (u_1, u_2) \in E(G) \wedge u_1 \in \nreach(2, v) \right\}
\end{align*}
\end{theorem}

\begin{proof}
For $k = 1$ the only sr-path of cost $1$ starting from $s$ is $\langle s \rangle$ and it contains no edges. Thus $\nreach(1, s) = \emptyset$.
For $k = 2$, a sr-path of cost $2$ starting at $s$ has either the form $\langle s, v \rangle$ where
$v \in V(G)$ or $\langle (s, v) \rangle$ where $(s, v) \in E(G)$. Thus, for each $v$ such that $\sp(s, v)$ is a path,
the last edge will belong to $\ereach(2, s)$. Also, every edge of the form $(s, v)$ will belong to $\ereach(2, s)$

Let $k \geq 3$.

$(\subseteq)$ Let $e \in \ereach(k, v)$. Then there exists a deterministic sr-path 
$\sr{p} = \langle x_1, \ldots, x_{l - 1}, x_l \rangle$  of segment cost at most $k$ such that
either $x_l = e$ or $x_l$ is a node
segment and $e$ is the last edge of the unique shortest path between $x^2_{l - 1}$
and $x_{l}$. Let $\sr{q} = \langle x_1, \ldots x_{l - 1} \rangle$ and $v = x^2_{l - 1}$.

\emph{Case 1:} $x_l = e = (u_1, u_2)$. Let 
Since $\cost(\sr{p}) \leq k$, we know that
$\cost(\sr{q}) \leq k - 2$. Thus $v = x^2_{l - 1} \in \nreach(k - 2, s)$. 
Since $\sr{p}$ is deterministic, 
there must be a unique shortest path
between $x^2_{l - 1} = v$ to $x^1_l = u_1$. Therefore, $u_1 \in \nreach(2, v)$ so
$$
e \in \bigcup_{v \in \nreach(k - 2, s)}  \left\{ (u_1, u_2) \mid (u_1, u_2) \in E(G) \wedge u_1 \in \nreach(2, v) \right\}.
$$

\emph{Case 2:} $x_l$ is a node segment and $e$ is the last edge of the unique
shortest path between $x^2_{l - 1}$ and $x_l$. Then $\cost(\sr{q}) \leq k - 1$
so that $v \in \nreach(s, k - 1)$. Since $e$ is the last edge of the unique
shortest path between $v = x^2_{l - 1}$ and $x_l$ we have that $e \in \ereach(2, v)$.
Thus
$$
e \in \bigcup_{v \in \nreach(k - 1, s)} \ereach(2, v).
$$

$(\supseteq)$ Let $e \in \bigcup_{v \in \nreach(k - 1, s)} \ereach(v, 2)$. Then
there exists $v \in \nreach(k - 1, s)$ such that $e \in \ereach(v, 2)$. By definition,
this means that there exists a deterministic sr-path $\sr{p} = \langle x_1, \ldots, x_l \rangle$
from $s$ to $v$ with $\cost(\sr{p}) \leq k - 1$ and a unique shortest path from $v$ to $u$
such that $e$ is the last edge of that path. Thus, $\langle x_1, \ldots, x_l, u \rangle$ is a deterministic
sr-path from $s$ to $u$ of cost at most $k - 1 + 1 = k$ with last edge $e$. We conclude that
$e \in \ereach(s, k)$.

Let $(u_1, u_2) \in \bigcup_{v \in \nreach(k - 2, s)}  \left\{ (u_1, u_2) \mid (u_1, u_2) \in E(G) \wedge u_1 \in \nreach(2, v) \right\}$.
Then there exists $v \in \nreach(k - 2, s)$ such that $u_1 \in \nreach(2, v)$. 
By definition, this means that we have a deterministic sr-path $\sr{p} = \langle x_1, \ldots, x_l \rangle$ from $s$ to $v$ with segment cost at most $k - 2$ and
a unique shortest path from $v$ to $u_1$. Therefore $\langle x_1, \ldots, x_l, (u_1, u_2) \rangle$ is a deterministic sr-path
from $s$ to $u_2$ of segment cost at most $k - 2 + 2 = k$ with last element being an adjacency segment
on $(u_1, u_2)$. Hence, $(u_1, u_2) \in \ereach(k, s)$.
\end{proof}


% \begin{proposition}
% Let $G$ be a network, $s \in V(G)$. Then for $k \geq 2$,
% \begin{align*}
% \nreach(0, s) = & \left\{ s \right\} \\
% \nreach(1, s) = & \left\{ V(\sp(s, v)) \mid \textrm{$v \in V(G)$ and $\sp(s, v)$ is a path} \right\} \\
% \nreach(k, s) = & \bigcup_{v \in \nreach(k - 1, s)} \nreach(1, v) \quad \cup \\
%                 & \left\{ u_2 \mid (u_1, u_2) \in E(G) \wedge \exists u \in \nreach(k - 2, s) : u_1 \in \nreach(1, u) \right\}
% \end{align*}
% \end{proposition}
% 
% \begin{proof}
% If $k = 0$ then the only sr-path starting at $s$ of segment cost at most $0$ is $\langle s \rangle$. Thus, since $V(\langle s \rangle) = s$, $\nreach(0, s) = \{ s \}$.
% For $k = 1$, the sr-paths starting at $s$ with segments cost at most $1$ are for the form $\langle s, v \rangle$ for $v \in V(G)$. From these, we only consider the ones
% that are deterministic, that is, the ones such that there is a unique shortest path between $s$ and $v$. In other words, the ones such that $\sp(s, v)$ is a path.
% 
% For $k \geq 2$ we will prove the that each set is contained inside the other. 
% 
% $(\subseteq)$ Let $v \in \nreach(k, s)$. Then, by definition, there exists a deterministic sr-path $\sr{p} = \langle x_1, \ldots, x_l \rangle$
% of cost at most $k$ such that $v \in V(\sr{p})$.
% \end{proof}
