\section{Disjoint paths and network flows}
\label{section:dp}

The problem of computing disjoint paths in a graph is one of those ubiquitous problem that has driven
a lot of research over the years. Numerous algorithms exist for solving it, most of them being some variant
of the more general maximum flow problem \cite{suurballe1984quick, Ahuja}.

The maximum flow problem is the problem of finding the maximum amount of information that can
be sent between two given nodes in a network. It is closely related to the multi-commodity flow
problem that we studied in Chapter \ref{chapter:te}. Instead of being given a set of demands, we are given
a source $s \in V(G)$ and a destination $t \in V(G)$ and we are asked what is the maximum amount of traffic that can be routed 
between those two nodes without exceeding the capacity of any edge. Another way to see it is to 
imagine that we have an infinite amount of unit demands between $s$ and $t$ and we are asked what is the maximum
number of such demands that can simultaneously be routed of the network without exceeding any link capacity.

The maximum flow problem admits an IP formulation that is very similar to the formulation that we gave for the MCF. If we let $x_e$
define the amount of demands routed over edge $e$ it can be shown that the following LP models the problem \cite{Ahuja}.

\begin{center}
\begin{tabular}{crcllr}
\multicolumn{5}{l}{$\maxflow(G, s, t)$} \\[0.5cm] 
$\displaystyle \mathbf{max}$ & $\displaystyle \sum_{e \in \oute(s)} x_e$ & & & & \\[0.5cm]
$\textbf{s.t.}$ & $\displaystyle \sum_{e \in \delta^-(v)} x_{e} - \sum_{e \in \delta^+(v)} x_{e}$ & $=$    & $ 0$ & $\forall v \in V(G) \setminus \{ s, t \}$  &  \\[0.5cm]
                & $x_{e}$ & $\leq$ & $\bnd(e)$ & $\forall e \in E(G)$ & \\[0.5cm]
                & $x_{e}$ & $\in$ & $\mathbb{N}$
\end{tabular}
\end{center}

It is possible to show that the linear programming relaxation of this problem obtained by replacing $x_e \in \mathbb{N}$ by $x_e \geq 0$ 
is equivalent as long as the capacities are integral \cite{Ahuja}. This makes the maximum flow problem one of those rare cases where the integrality constraints no not increase the difficulty
of the problem. Numerous polynomial time algorithms have been developed for solving the maximum flow problem \cite{Edmonds:1972:TIA:321694.321699, dinitz, Goldberg:1988:NAM:48014.61051, Ahuja}.
If we set \emph{unit capacities} on the edges and think about the maximum flow problem as one answering the question of what is the maximum amount of unit demands that we can
route from $s$ to $t$, we see that we actually end up with the \emph{maximum number of disjoint path between $s$ and $t$}. This is the case
since each of the routed demands must follow a path that shares no edges with any of the other demand paths or otherwise some edge would carry at least two units
of traffic, thus exceeding its capacity.

In other words, this means that we can model the problem of computing the maximum number of edge-disjoint paths between $s$ and $t$ by replacing $\bnd(e)$
by $1$ in the $\maxflow$ model. By doing so, we obtain the following model which can also be solved efficiently.

\begin{center}
\begin{tabular}{crcllr}
\multicolumn{5}{l}{$\maxedp(G, s, t)$} \\[0.5cm] 
$\displaystyle \mathbf{max}$ & $\displaystyle \sum_{e \in \oute(s)} x_e$ & & & & \\[0.5cm]
$\textbf{s.t.}$ & $\displaystyle \sum_{e \in \delta^-(v)} x_{e} - \sum_{e \in \delta^+(v)} x_{e}$ & $=$    & $ 0$ & $\forall v \in V(G) \setminus \{ s, t \}$  &  \\[0.5cm]
                & $x_{e}$ & $\leq$ & $1$ & $\forall e \in E(G)$ & \\[0.5cm]
                & $x_{e}$ & $\in$ & $\mathbb{N}$
\end{tabular}
\end{center}

This model focuses solely one providing a maximum amount of paths. It does not care about the quality of those paths, for instance, in terms of latency. It turns out that if we wish to
minimize total latency of the path set we can still do it in polynomial time. This problem is known, in general, as the minimum cost maximum flow problem \cite{Ahuja}.
We can easily adapt the above model to obtain a set of $n$ disjoint paths (if they exist) whose total latency is as small as possible by
requiring the total flow out of $s$ to be $n$ and minimizing the sum of the latencies of the edges with a non-zero flow as shown in the following model.

\begin{center}
\begin{tabular}{crcllr}
\multicolumn{5}{l}{$\minlatedp(G, s, t)$} \\[0.5cm] 
$\displaystyle \mathbf{min}$ & $\displaystyle \sum_{e \in E(G)} \lat(e) \cdot x_e$ & & & & \\[0.5cm]
$\textbf{s.t.}$ & $\displaystyle \sum_{e \in \delta^-(v)} x_{e} - \sum_{e \in \delta^+(v)} x_{e}$ & $=$    & $ 0$ & $\forall v \in V(G) \setminus \{ s, t \}$  &  \\[0.5cm]
                & $\displaystyle \sum_{e \in \oute(s)} x_e$ & $=$ & $n$ & \\[0.5cm]
                & $x_{e}$ & $\leq$ & $1$ & $\forall e \in E(G)$ & \\[0.5cm]
                & $x_{e}$ & $\in$ & $\mathbb{N}$
\end{tabular}
\end{center}

If we want $n$ to be as large as possible we can simply first compute the optimal solution of $\maxedp(G, s, t)$ and set $n$ to that value.
Surprisingly, this problem can also be solved in polynomial time \cite{Ahuja}. Unfortunately, slight variations of this problem that are of interest for computer
networks quickly become \NPhard. For instance, if we want to minimize the maximum latency over all the paths rather than the latency sum, the problem is 
\NPhard~even if $n = 2$ \cite{minmax-disjoint-90, Li1990}. Another change that, perhaps surprisingly, makes the problem \NPhard~is to have two origins $s_1, s_2$ and two destinations
$t_1, t_2$ and ask for a pair of disjoint paths, one from $s_1$ to $t_1$ and another from $s_2$ to $t_2$ \cite{VYGEN199583}. One might naively think that we could get away by adding a fake node 
$s$ linked to both $s_1, s_2$ and link $t_1, t_2$ to some other fake $t$ and then compute the maximum flow from $s$ to $t$. Unfortunately, nothing in the model
forces the flow originating from $s_1$ to go to $t_1$ instead of $t_2$ (and vice-versa). In order to force some origin to destination assignment, we cannot rely on a
maximum flow model but rather on a multi-commodity flow model which, as we have already seen, is harder to solve. 
