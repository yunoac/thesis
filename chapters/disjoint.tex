\chapter{Disjoint paths with SR}
\label{chapter:disjoint}

\section*{Introduction}

For an Internet Service Provider (ISP), providing disjoint paths to its
customers might be one of those advanced connectivity services that generate
more revenues. This is what emerged from our discussion with a national ISP that
we call ``reference ISP''. When we met them, this ISP's operators themselves
steered the discussion towards possibilities to provide disjoint paths between
sites to which a customer is connected. They were motivated by requests from
banks and financial customers.

We have quickly realized that the disjoint-path connectivity service has a much
bigger market than our reference ISP. An illustration is provided by the
NANOG email discussion about a major outage of the Bell network on August
4th, 2017~\cite{refnanog}. The email thread started with Bell's customers
complaining that both Internet and mobile connectivity were completely absent in
East Canada, affecting banking, ATM, land lines and even 911 services. When a
single fibre cut was indicated as the cause of the outage, someone expressed
doubts that ISPs really provide geographically diverse circuits, irrespectively
of what they promise and sell. The following emails discussed the impossibility
to work around this limitation by relying on two providers, as their networks
may share the same physical infrastructure (fibres, conduit, etc.), without the
ISPs even knowing it -- as they do not share information between each other.

The discussion we had with the reference ISP's operators was indeed focused on
\textit{providing disjoint paths within a single ISP}, their own. A possible
solution~\cite{art:2014} to achieve this goal is to deploy two parallel
networks, say a red and a blue copy of the same topology, and configure the
intra-domain routing protocol (IGP) so that any packet is forwarded in only one of
the two networks -- i.e., packets that enter the red copy are only forwarded in
the red copy.
The few links between the two networks are only used if one of the two copies is
partitioned. This architecture provides disjoint paths by design, but it is very
expensive since the entire network is doubled.
The reference ISP's operators were therefore reluctant to deploy it.
%The operators discarded solutions
%based on adding physical redundancy, like the 1+1 protection scheme in optical
%networks~\cite{}, because of the additional costs. 
Of course, they were also aware that MPLS tunnels can be created over
arbitrary paths with RSVP-TE~\cite{rfc3209}, including disjoint ones, on an
existing infrastructure.
However, they were in the process of moving away from MPLS, in order to avoid
its operational limitations~\cite{mpls-opissues-ripe64}, its sub-optimal usage
of resources~\cite{mpls-latency-imc11} and its scalability challenges with
respect to the routers' state~\cite{rfc5439,defo-sig15}.

%including synchronization with the
%underlying IGP and control-plane overhead~\cite{}, scalability~\cite{} and
%latency inflation~\cite{mpls-latency-imc11}.

Looking at other ISPs, our operators were instead considering Segment Routing
(SR). Motivated by this, we dedicate this chapter to the study of the problem of computing and implementing disjoint paths over a network
with segment routing. We also provide a solution for leveraging some properties of sr-paths to
show how we can provide disjoint paths that are robust to link failures.


\input{chapters/disjoint/dp.tex}

\input{chapters/disjoint/dpsr.tex}

\input{chapters/disjoint/rdp.tex}

