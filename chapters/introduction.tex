\chapter{Introduction}
\label{chapter:into}

This thesis aims at providing a first mathematical formalization of segment routing and 
showcase several of its use cases. We believe that such a formalization is
going to help advance the research of the algorithmic aspects of segment routing by
providing a solid mathematical foundation and notations which can serve as a starting
point for others to research this topic.
The presented use cases validate the practical interest of the deployment of segment routing
on existing networks.

Traditional networks use shortest path routing to forward the packets between the routers
in the network. Without going into details of shortest path routing, each link in the network 
has a weight configured on it and shortest paths are computed with respect to these weights. We refer to these weights
as interior gateway protocol weights (IGP weights). These paths are
used to create a routing table on each router that maps the destination of the packet to an outgoing link interface.
In this way, when a packet to $dst$ reaches a router, that router will inspect its routing table to find out
the link interface towards which it needs to forward that packet. By doing so, the packet will reach $dst$ by traversing the
shortest path relative to these IGP weights. 
Figure \ref{fig:intro_sprouting} provides a high level illustration of shortest path routing of the Abilene network \cite{abilene}.
When forwarding packets from router \node{a} to \node{i}, the shortest path, shown in green, will be traversed.
The table on top of router \node{d} shows an abstraction of its routing
table. When \node{d} receives a packet with destination \node{i}, it inspects its routing table and sees that
the next hop in the IGP shortest path to \node{i} is \node{e} so it forwards it there.
All of this is an abstraction of what really goes on in a real network because it hides a lot of technical 
aspects like, for instance, the fact that the information stored is the IP address. However, for this thesis, this view is accurate enough as we focus on mathematical optimization problems related to
computer networks rather than compute networks per se.

\begin{figure}[H]
\begin{center}
\begin{tikzpicture}
\def\x{0}
\def\y{0}
\node[scale=0.15] (a) at (0 + \x,  0 + \y) {\router{a}{green}};
\node[scale=0.15] (b) at (-1 + \x,  -2 + \y) {\router{b}{router}};
\node[scale=0.15] (c) at (0 + \x,  -3.5 + \y) {\router{c}{router}};
\node[scale=0.15] (d) at (2.5 + \x,  -1.5 + \y) {\router{d}{router}};
\node[scale=0.15] (e) at (4.5 + \x,  -2 + \y) {\router{e}{router}};
\node[scale=0.15] (f) at (4 + \x,  -4 + \y) {\router{f}{router}};
\node[scale=0.15] (i) at (7 + \x,  -3.5 + \y) {\router{i}{green}};
\node[scale=0.15] (h) at (7 + \x,  -1.8 + \y) {\router{h}{router}};
\node[scale=0.15] (g) at (5.8 + \x,  -0.5 + \y) {\router{g}{router}};
\node[scale=0.15] (k) at (9.5 + \x,  -2 + \y) {\router{k}{router}};
\node[scale=0.15] (j) at (10 + \x,  0 + \y) {\router{j}{router}};
\draw[line width=2] (a) edge[above, sloped] node[black,font=\bfseries] {\footnotesize \texttt{42}} (b);
\draw[line width=2] (b) edge[below, sloped] node[black,font=\bfseries] {\footnotesize \texttt{46}} (c);
\draw[line width=2] (b) edge[below, sloped] node[black,font=\bfseries] {\footnotesize \texttt{34}} (d);
\draw[line width=2] (a) edge[below, sloped] node[black,font=\bfseries] {\footnotesize \texttt{31}} (d);
\draw[line width=2] (c) edge[below, sloped] node[black,font=\bfseries] {\footnotesize \texttt{35}} (f);
\draw[line width=2] (d) edge[below, sloped] node[black,font=\bfseries] {\footnotesize \texttt{43}} (e);
\draw[line width=2] (e) edge[above, sloped] node[black,font=\bfseries] {\footnotesize \texttt{27}} (f);
\draw[line width=2] (e) edge[above, sloped] node[black,font=\bfseries] {\footnotesize \texttt{21}} (g);
\draw[line width=2] (g) edge[above, sloped] node[black,font=\bfseries] {\footnotesize \texttt{11}} (h);
\draw[line width=2] (h) edge[below, sloped] node[black,font=\bfseries] {\footnotesize \texttt{27}} (i);
\draw[line width=2] (f) edge[below, sloped] node[black,font=\bfseries] {\footnotesize \texttt{30}} (i);
\draw[line width=2] (g) edge[above, sloped] node[black,font=\bfseries] {\footnotesize \texttt{61}} (j);
\draw[line width=2] (j) edge[above, sloped] node[black,font=\bfseries] {\footnotesize \texttt{52}} (k);
\draw[line width=2] (k) edge[above, sloped] node[black,font=\bfseries] {\footnotesize \texttt{42}} (i);
\draw[line width=3, darkgreen] (a) edge[->] (d);
\draw[line width=3, darkgreen] (d) edge[->] (e);
\draw[line width=3, darkgreen] (e) edge[->] (f);
\draw[line width=3, darkgreen] (f) edge[->] (i);
%\draw[darkgreen, line width=3, ->] plot [smooth] coordinates { ($(a)+(0.6,0.1)$) ($(d)+(0,0.5)$) ($(e)+(0,0.6)$) ($(e)+(0.6,0)$) ($(f)+(0.5,0.5)$) ($(i)+(-0.5,0.25)$) };


\node (t) at (\x + 3.5, \y + 1) {
\begin{tabular}{|c|c|}
\hline
dest & next hop \\
\hline
$\vdots$ & $\vdots$ \\
\hline
\node{c} & \node{b} \\
\hline
\rowcolor{green}
\node{i} & \node{e} \\
\hline
\node{j} & \node{e} \\
\hline
$\vdots$ & $\vdots$ \\
\hline
\end{tabular}
};

\draw[line width=1, dotted] (d) -- (t);
\end{tikzpicture}
\end{center}
\caption{Illustration of shortest path routing on the Abilene network.}
\label{fig:intro_sprouting}
\end{figure}

During recent years, networking vendors and network operators have
designed \cite{filsfils2015segment,rfc8402}, implemented
\cite{srdemo} and deployed \cite{schuller2017traffic,srdeploy} 
a new routing architecture called Segment Routing (SR). Segment
Routing is a modern variant of source routing which can be used in
either MPLS or IPv6 networks. In a nutshell, Segment Routing allows
the source of a packet or the ingress node in a network to easily
specify the path that packets need to follow inside the
network. This path is specified as a series of labels, called \emph{segments}, (MPLS labels or
IPv6 addresses in a special IPv6 header extension) that are added to
each packet. These segment are stored as a stack and are popped out as the packets reaches
the routers or links they represent. Some implementations do not explicitly remove the segments
but rather keep a point to the next segment.

More concretely, a segment routing path is composed of one or more segments. 
There are two types of segments: $(i)$ \emph{node} and $(ii)$ \emph{adjacency} segments. 
Node segments represent routers. When a node segment is at the top of the stack, the packet is routed towards the corresponding router using
standard shortest path routing. Figure \ref{fig:srnode} shows an example over the Abilene network of what happens when packets are
sent from \node{a} to \node{i} using segment routing with one intermediate node segment representing node \node{g}. First, \node{a} will inspect the
segment stack and see a node segment representing router \node{g}. It will pop this segment and forward the packet using
shortest path routing to \node{g}. Then, node \node{g} will inspect the packet header and see that the next segment is router \node{i}.
It will pop this segment and forward the packet using shortest path routing to \node{i}. At this point the segment stack is empty so the
packet as arrived to its destination.

\begin{figure}[H]
\begin{center}
\begin{tikzpicture}
\def\x{0}
\def\y{0}
\node[scale=0.15] (a) at (0 + \x,  0 + \y) {\router{a}{green}};
\node[scale=0.15] (b) at (-1 + \x,  -2 + \y) {\router{b}{router}};
\node[scale=0.15] (c) at (0 + \x,  -3.5 + \y) {\router{c}{router}};
\node[scale=0.15] (d) at (2.5 + \x,  -1.5 + \y) {\router{d}{router}};
\node[scale=0.15] (e) at (4.5 + \x,  -2 + \y) {\router{e}{router}};
\node[scale=0.15] (f) at (4 + \x,  -4 + \y) {\router{f}{router}};
\node[scale=0.15] (i) at (7 + \x,  -3.5 + \y) {\router{i}{green}};
\node[scale=0.15] (h) at (7 + \x,  -1.8 + \y) {\router{h}{router}};
\node[scale=0.15] (g) at (5.8 + \x,  -0.5 + \y) {\router{g}{green}};
\node[scale=0.15] (k) at (9.5 + \x,  -2 + \y) {\router{k}{router}};
\node[scale=0.15] (j) at (10 + \x,  0 + \y) {\router{j}{router}};
\draw[line width=2] (a) edge[above, sloped] node[black,font=\bfseries] {\footnotesize \texttt{42}} (b);
\draw[line width=2] (b) edge[below, sloped] node[black,font=\bfseries] {\footnotesize \texttt{46}} (c);
\draw[line width=2] (b) edge[below, sloped] node[black,font=\bfseries] {\footnotesize \texttt{34}} (d);
\draw[line width=2] (a) edge[below, sloped] node[black,font=\bfseries] {\footnotesize \texttt{31}} (d);
\draw[line width=2] (c) edge[below, sloped] node[black,font=\bfseries] {\footnotesize \texttt{35}} (f);
\draw[line width=2] (d) edge[below, sloped] node[black,font=\bfseries] {\footnotesize \texttt{43}} (e);
\draw[line width=2] (e) edge[below, sloped] node[black,font=\bfseries] {\footnotesize \texttt{27}} (f);
\draw[line width=2] (e) edge[above, sloped] node[black,font=\bfseries] {\footnotesize \texttt{21}} (g);
\draw[line width=2] (g) edge[above, sloped] node[black,font=\bfseries] {\footnotesize \texttt{11}} (h);
\draw[line width=2] (h) edge[above, sloped] node[black,font=\bfseries] {\footnotesize \texttt{27}} (i);
\draw[line width=2] (f) edge[above, sloped] node[black,font=\bfseries] {\footnotesize \texttt{35}} (i);
\draw[line width=2] (g) edge[above, sloped] node[black,font=\bfseries] {\footnotesize \texttt{61}} (j);
\draw[line width=2] (j) edge[above, sloped] node[black,font=\bfseries] {\footnotesize \texttt{52}} (k);
\draw[line width=2] (k) edge[above, sloped] node[black,font=\bfseries] {\footnotesize \texttt{42}} (i);
\draw[line width=3, darkgreen] (a) edge[->] (d);
\draw[line width=3, darkgreen] (d) edge[->] (e);
\draw[line width=3, darkgreen] (e) edge[->] (g);

\draw[line width=3, darkgreen] (g) edge[->] (h);
\draw[line width=3, darkgreen] (h) edge[->] (i);

\node[above of=a] (s1) {
\begin{tabular}{|c|c|}
\hline
\node{g} & \node{i} \\
\hline
\end{tabular}
};

\draw[line width=1, dotted] (a) -- (s1);

\node[above of=g] (s2) {
\begin{tabular}{|c|}
\hline
\node{i} \\
\hline
\end{tabular}
};

\draw[line width=1, dotted] (g) -- (s2);
\end{tikzpicture}
\end{center}
\caption{Illustration of SR from \node{a} to \node{i} with a node segment on \node{g}.}
\label{fig:srnode}
\end{figure}

Adjacency segments represent links. When an adjacency segment is at the top of the stack, the packet is routed towards the origin of the edge
corresponding to the link using standard shortest path routing. Then, it is forwarded over that specific link to the router corresponding
to the destination of that edge.

Figure \ref{fig:sradj} illustrates an example of what happens when packets are
sent from \node{a} to \node{i} using segment routing with an adjacency segment representing edge $(\node{g}, \node{j})$. First, \node{a} will inspect the
segment stack and see an adjacency segment representing edge $(\node{g}, \node{j})$. It will forward the packet using
shortest path routing to \node{g}. Then, node \node{g} will inspect the packet header and see that there is an adjacency segment of which he is the origin.
It pops the adjacency segment from the segment stack and sends it over edge $(\node{g}, \node{j})$. From there,
\node{j} will inspect the segment stack and see that in needs to forward that packet towards \node{i}.

\begin{figure}[H]
\begin{center}
\begin{tikzpicture}
\def\x{0}
\def\y{0}
\node[scale=0.15] (a) at (0 + \x,  0 + \y) {\router{a}{green}};
\node[scale=0.15] (b) at (-1 + \x,  -2 + \y) {\router{b}{router}};
\node[scale=0.15] (c) at (0 + \x,  -3.5 + \y) {\router{c}{router}};
\node[scale=0.15] (d) at (2.5 + \x,  -1.5 + \y) {\router{d}{router}};
\node[scale=0.15] (e) at (4.5 + \x,  -2 + \y) {\router{e}{router}};
\node[scale=0.15] (f) at (4 + \x,  -4 + \y) {\router{f}{router}};
\node[scale=0.15] (i) at (7 + \x,  -3.5 + \y) {\router{i}{green}};
\node[scale=0.15] (h) at (7 + \x,  -1.8 + \y) {\router{h}{router}};
\node[scale=0.15] (g) at (5.8 + \x,  -0.5 + \y) {\router{g}{green}};
\node[scale=0.15] (k) at (9.5 + \x,  -2 + \y) {\router{k}{router}};
\node[scale=0.15] (j) at (10 + \x,  0 + \y) {\router{j}{green}};
\draw[line width=2] (a) edge[above, sloped] node[black,font=\bfseries] {\footnotesize \texttt{42}} (b);
\draw[line width=2] (b) edge[below, sloped] node[black,font=\bfseries] {\footnotesize \texttt{46}} (c);
\draw[line width=2] (b) edge[below, sloped] node[black,font=\bfseries] {\footnotesize \texttt{34}} (d);
\draw[line width=2] (a) edge[below, sloped] node[black,font=\bfseries] {\footnotesize \texttt{31}} (d);
\draw[line width=2] (c) edge[below, sloped] node[black,font=\bfseries] {\footnotesize \texttt{35}} (f);
\draw[line width=2] (d) edge[below, sloped] node[black,font=\bfseries] {\footnotesize \texttt{43}} (e);
\draw[line width=2] (e) edge[below, sloped] node[black,font=\bfseries] {\footnotesize \texttt{27}} (f);
\draw[line width=2] (e) edge[above, sloped] node[black,font=\bfseries] {\footnotesize \texttt{21}} (g);
\draw[line width=2] (g) edge[above, sloped] node[black,font=\bfseries] {\footnotesize \texttt{11}} (h);
\draw[line width=2] (h) edge[above, sloped] node[black,font=\bfseries] {\footnotesize \texttt{27}} (i);
\draw[line width=2] (f) edge[above, sloped] node[black,font=\bfseries] {\footnotesize \texttt{35}} (i);
\draw[line width=2] (g) edge[above, sloped] node[black,font=\bfseries] {\footnotesize \texttt{61}} (j);
\draw[line width=2] (j) edge[above, sloped] node[black,font=\bfseries] {\footnotesize \texttt{52}} (k);
\draw[line width=2] (k) edge[above, sloped] node[black,font=\bfseries] {\footnotesize \texttt{42}} (i);
\draw[line width=3, darkgreen] (a) edge[->] (d);
\draw[line width=3, darkgreen] (d) edge[->] (e);
\draw[line width=3, darkgreen] (e) edge[->] (g);
\draw[line width=4, dotted, darkgreen] (g) edge[->] (j);

\draw[line width=3, darkgreen] (j) edge[->] (k);
\draw[line width=3, darkgreen] (k) edge[->] (i);

\node[above of=a] (s1) {
\begin{tabular}{|c|c|}
\hline
$(\node{g}, \node{j})$ & \node{i} \\
\hline
\end{tabular}
};

\draw[line width=1, dotted] (a) -- (s1);

\node[above of=g] (s2) {
\begin{tabular}{|c|c|}
\hline
$(\node{g}, \node{j})$ & \node{i} \\
\hline
\end{tabular}
};


\node[above of=j] (s3) {
\begin{tabular}{|c|}
\hline
\node{i} \\
\hline
\end{tabular}
};

\draw[line width=1, dotted] (g) -- (s2);
\end{tikzpicture}
\end{center}
\caption{Illustration of SR from \node{a} to \node{i} with an adjacency segment on $(\node{g}, \node{j})$.}
\label{fig:sradj}
\end{figure}

\section{Why segment routing}

Segments routing offers a lot of flexibility for routing packets by allowing to exploit non shortest path as illustrated
in Figures \ref{fig:srnode} and \ref{fig:sradj}.
It does so by leveraging traditional IP routing so it is easily implementable on current networks without
needing to change the way networks operate or their infrastructure. Moreover, there is no need for routers
to maintain more state than their, already existing, routing tables. The only overhead is on the packet header
which now needs to contain the segment stack.

SR is not the first technique to have been developed that makes it possible to route over non shortest paths. 
Another technique is to use Multiprotocol Label Switching (MPLS) \cite{Ghein:2006:MF:1206881}. In a high level, MPLS works by
configuring label forwarding tables on the routers. Conceptually, these tables are similar to the routing tables except that
they are not built with respect to the IGP weights. Instead, they map pairs of (label, incoming interface) into other pairs (label, outgoing interface). 
Figure \ref{fig:mpls} shows an example of how a MPLS configuration can allow us to route packets from
\node{a} to \node{c} over the non-shortest path $(\node{a}, \node{d}, \node{b}, \node{c})$. In this example, this is achieved by configuring the labeling table of \node{d} to map 
packets coming from the interface corresponding to link $(\node{a}, \node{d})$ with label $5$ to label $8$
and link $(\node{d}, \node{b})$. Second, we configure the labeling
table of \node{b} to map packets coming from link $(\node{d}, \node{b})$ with label $8$ to be routed over $(\node{b}, \node{c})$
with the same label.

\begin{figure}[H]
\begin{center}
\begin{tikzpicture}
\def\x{0}
\def\y{0}
\node[scale=0.15] (a) at (0 + \x,  0 + \y) {\router{a}{green}};
\node[scale=0.15] (b) at (-1 + \x,  -2 + \y) {\router{b}{green}};
\node[scale=0.15] (c) at (0 + \x,  -3.5 + \y) {\router{c}{router}};
\node[scale=0.15] (d) at (2.5 + \x,  -1.5 + \y) {\router{d}{router}};
\node[scale=0.15] (e) at (4.5 + \x,  -2 + \y) {\router{e}{router}};
\node[scale=0.15] (f) at (4 + \x,  -4 + \y) {\router{f}{router}};
\node[scale=0.15] (i) at (7 + \x,  -3.5 + \y) {\router{i}{router}};
\node[scale=0.15] (h) at (7 + \x,  -1.8 + \y) {\router{h}{router}};
\node[scale=0.15] (g) at (5.8 + \x,  -0.5 + \y) {\router{g}{router}};
\node[scale=0.15] (k) at (9.5 + \x,  -2 + \y) {\router{k}{router}};
\node[scale=0.15] (j) at (10 + \x,  0 + \y) {\router{j}{router}};
\draw[line width=2] (a) edge[above, sloped] node[black,font=\bfseries] {\footnotesize \texttt{42}} (b);
\draw[line width=2] (b) edge[below, sloped] node[black,font=\bfseries] {\footnotesize \texttt{46}} (c);
\draw[line width=2] (b) edge[below, sloped] node[black,font=\bfseries] {\footnotesize \texttt{34}} (d);
\draw[line width=2] (a) edge[below, sloped] node[black,font=\bfseries] {\footnotesize \texttt{31}} (d);
\draw[line width=2] (c) edge[below, sloped] node[black,font=\bfseries] {\footnotesize \texttt{35}} (f);
\draw[line width=2] (d) edge[below, sloped] node[black,font=\bfseries] {\footnotesize \texttt{43}} (e);
\draw[line width=2] (e) edge[below, sloped] node[black,font=\bfseries] {\footnotesize \texttt{27}} (f);
\draw[line width=2] (e) edge[above, sloped] node[black,font=\bfseries] {\footnotesize \texttt{21}} (g);
\draw[line width=2] (g) edge[above, sloped] node[black,font=\bfseries] {\footnotesize \texttt{11}} (h);
\draw[line width=2] (h) edge[above, sloped] node[black,font=\bfseries] {\footnotesize \texttt{27}} (i);
\draw[line width=2] (f) edge[above, sloped] node[black,font=\bfseries] {\footnotesize \texttt{35}} (i);
\draw[line width=2] (g) edge[above, sloped] node[black,font=\bfseries] {\footnotesize \texttt{61}} (j);
\draw[line width=2] (j) edge[above, sloped] node[black,font=\bfseries] {\footnotesize \texttt{52}} (k);
\draw[line width=2] (k) edge[above, sloped] node[black,font=\bfseries] {\footnotesize \texttt{42}} (i);

\draw[line width=3, darkgreen] (a) edge[->] (d);
\draw[line width=3, darkgreen] (d) edge[->] (b);
\draw[line width=3, darkgreen] (b) edge[->] (c);


\node (p1) at (\x + 2, \y + -0.25) {
\begin{tabular}{|c|c|}
\hline
\footnotesize 5 & \footnotesize data \\
\hline
\end{tabular}
};

\draw[line width=1, dashed] (p1) -- (1.5, -0.8);


\node (p2) at (\x + 2.5, \y + -2.3) {
\begin{tabular}{|c|c|}
\hline
\footnotesize 8 & \footnotesize data \\
\hline
\end{tabular}
};

\draw[line width=1, dashed] (p2) -- (1.5, -1.8);


\node (t) at (\x + 5, \y + 1.5) {
\begin{tabular}{|c|c|c|c|}
\hline
lbl in & in inter & lbl out & out inter \\
\hline
$\vdots$ & $\vdots$ & $\vdots$ & $\vdots$ \\
\hline
\rowcolor{green}
5 & (\node{a}, \node{d}) & 8 & (\node{d}, \node{b}) \\
\hline
$\vdots$ & $\vdots$ & $\vdots$ & $\vdots$ \\
\hline
\end{tabular}
};

\draw[line width=1, dotted] (d) -- (t);

\node (t2) at (\x + 2, \y + -6) {
\begin{tabular}{|c|c|c|c|}
\hline
lbl in & in inter & lbl out & out inter \\
\hline
$\vdots$ & $\vdots$ & $\vdots$ & $\vdots$ \\
\hline
\rowcolor{green}
8 & (\node{d}, \node{b}) & 8 & (\node{b}, \node{c}) \\
\hline
$\vdots$ & $\vdots$ & $\vdots$ & $\vdots$ \\
\hline
\end{tabular}
};

\draw[line width=1, dotted] (b) -- (-0.75, -4.8);

\node (p3) at (\x + 1.5, \y + -2.3-0.75) {
\begin{tabular}{|c|c|}
\hline
\footnotesize 8 & \footnotesize data \\
\hline
\end{tabular}
};

\draw[line width=1, dashed] (p3) -- (-0.5, -2.5);

\end{tikzpicture}
\end{center}
\caption{Illustration of MPLS.}
\label{fig:mpls}
\end{figure}

One drawback of MPLS when compared to segment routing is that the routers need to maintain an extra state,
namely, the labeling tables. This makes it harder for solutions based on MPLS to scale has these
labeling tables can become huge if a lot of different paths are configured~\cite{rfc5439,defo-sig15}. 
MPLS also has some operational limitations \cite{mpls-opissues-ripe64} and makes a sub-optimal usage
of resources~\cite{mpls-latency-imc11}.

In contrast, as we mentioned above, with segment
routing only the ingress node needs to maintain a state and all the path information is inserted in 
the packed header as segment. However, one drawback of SR is that, due to hardware limitations, some commercial routers only support a very limited amount
of segments \cite{Tantsura_SID:2017}. An average low end router can support up to 5 segments whereas high end router can go up to 
about 10 segments.

Other more recent techniques exist that overcome these limitations as, for instance, Fibbing \cite{Vissicchio:2014:SLL:2670518.2673868}.
Fibbing is an new architecture proposed in  2014 that enables central control over distributed routing. By doing so, similarly to segment routing, it 
combines the advantages of software defined networks (flexibility, expressiveness, and manageability) and traditional approaches (robustness, and scalability).

The way Fibbing works is that it introduces fake nodes and links into an underlying link-state routing protocol. The fake routers and links
are then taken into account by routers when computing their forwarding tables. Figure \ref{fig:intro_fibbing} illustrates how to implement
the non shortest path $(\node{a}, \node{d}, \node{b}, \node{c})$ with fibbing. To do so, we can introduce a fake node \node{x} connected to \node{a} and \node{d}.
This fake node advertises that it can reach \node{c} directly, with a weight of 1. Therefore, from the point of view of \node{a}, the next hop
on the shortest path to \node{c} is not \node{b} but rather \node{x}. We ensure that the table are configured in a way such that when \node{a}
sends packets towards fake node \node{x}, link $(\node{a}, \node{d})$ is used. In this way, \node{a} will send the packets to \node{d}
and then from there routing will be done normally using shortest path $(\node{d}, \node{b}, \node{c})$. 


\begin{figure}[H]
\begin{center}
\begin{tikzpicture}
\def\x{0}
\def\y{0}
\node[scale=0.15] (a) at (0 + \x,  0 + \y) {\router{a}{green}};
\node[scale=0.15] (b) at (-1 + \x,  -2 + \y) {\router{b}{router}};
\node[scale=0.15] (c) at (0 + \x,  -3.5 + \y) {\router{c}{router}};
\node[scale=0.15] (d) at (2.5 + \x,  -1.5 + \y) {\router{d}{router}};
\node[scale=0.15] (e) at (4.5 + \x,  -2 + \y) {\router{e}{router}};
\node[scale=0.15] (f) at (4 + \x,  -4 + \y) {\router{f}{router}};
\node[scale=0.15] (i) at (7 + \x,  -3.5 + \y) {\router{i}{router}};
\node[scale=0.15] (h) at (7 + \x,  -1.8 + \y) {\router{h}{router}};
\node[scale=0.15] (g) at (5.8 + \x,  -0.5 + \y) {\router{g}{router}};
\node[scale=0.15] (k) at (9.5 + \x,  -2 + \y) {\router{k}{router}};
\node[scale=0.15] (j) at (10 + \x,  0 + \y) {\router{j}{router}};

\node[draw, fill=marked] (x) at (0.5 + \x,  -1 + \y) {\footnotesize \texttt{x}};

\draw[line width=2] (a) edge[below, sloped] node[black,font=\bfseries] {\footnotesize \texttt{1}} (x);

\draw[line width=2] (a) edge[above, sloped] node[black,font=\bfseries] {\footnotesize \texttt{42}} (b);
\draw[line width=2] (b) edge[below, sloped] node[black,font=\bfseries] {\footnotesize \texttt{46}} (c);
\draw[line width=2] (b) edge[below, sloped] node[black,font=\bfseries] {\footnotesize \texttt{34}} (d);
\draw[line width=2] (a) edge[above, sloped] node[black,font=\bfseries] {\footnotesize \texttt{31}} (d);
\draw[line width=2] (c) edge[below, sloped] node[black,font=\bfseries] {\footnotesize \texttt{35}} (f);
\draw[line width=2] (d) edge[below, sloped] node[black,font=\bfseries] {\footnotesize \texttt{43}} (e);
\draw[line width=2] (e) edge[above, sloped] node[black,font=\bfseries] {\footnotesize \texttt{27}} (f);
\draw[line width=2] (e) edge[above, sloped] node[black,font=\bfseries] {\footnotesize \texttt{21}} (g);
\draw[line width=2] (g) edge[above, sloped] node[black,font=\bfseries] {\footnotesize \texttt{11}} (h);
\draw[line width=2] (h) edge[below, sloped] node[black,font=\bfseries] {\footnotesize \texttt{27}} (i);
\draw[line width=2] (f) edge[below, sloped] node[black,font=\bfseries] {\footnotesize \texttt{30}} (i);
\draw[line width=2] (g) edge[above, sloped] node[black,font=\bfseries] {\footnotesize \texttt{61}} (j);
\draw[line width=2] (j) edge[above, sloped] node[black,font=\bfseries] {\footnotesize \texttt{52}} (k);
\draw[line width=2] (k) edge[above, sloped] node[black,font=\bfseries] {\footnotesize \texttt{42}} (i);


\draw[darkgreen, line width=3, ->] plot [smooth] coordinates { ($(a)+(0.3,-0.3)$) ($(x)+(0,0.3)$) ($(1.25, -0.75)$) ($(d)+(-0.5,0.2)$) };

\draw[line width=2, dotted] (x) edge[above, sloped] node {\footnotesize \texttt{1}} (c);

%\draw[line width=3, darkgreen] (a) edge[->] (d);
\draw[line width=3, darkgreen] (d) edge[->] (b);
\draw[line width=3, darkgreen] (b) edge[->] (c);

\draw[->, very thick, orange] (x) -- (1.25, -0.75);



%\draw[darkgreen, line width=3, ->] plot [smooth] coordinates { ($(a)+(0.6,0.1)$) ($(d)+(0,0.5)$) ($(e)+(0,0.6)$) ($(e)+(0.6,0)$) ($(f)+(0.5,0.5)$) ($(i)+(-0.5,0.25)$) };
% \node (t) at (\x + 3.5, \y + 1) {
% \begin{tabular}{|c|c|}
% \hline
% dest & next hop \\
% \hline
% $\vdots$ & $\vdots$ \\
% \hline
% \node{c} & \node{b} \\
% \hline
% \rowcolor{green}
% \node{i} & \node{e} \\
% \hline
% \node{j} & \node{e} \\
% \hline
% $\vdots$ & $\vdots$ \\
% \hline
% \end{tabular}
% };
%\draw[line width=1, dotted] (d) -- (t);
\end{tikzpicture}
\end{center}
\caption{Illustration of fibbing on the Abilene network.}
\label{fig:intro_fibbing}
\end{figure}

There is no particular reason for having chosen segment routing over fibbing other than the fact that we already had started 
this thesis when we first learned about fibbing. It would be interesting to make a comparative
study between the two technologies and better understand their limitations and when is one more suitable than the other.

\section{SR applications and contributions}

In what follows, we give an overview of the uses cases of segment routing that we study in this thesis.

\subsubsection*{Traffic engineering}

One of most successful and widely studied application of SR is traffic engineering (TE).
In a nutshell, TE is the problem of routing a set of demands in the most efficient way (avoiding congestion) on a given
network. A demand corresponds to a volume of traffic to be routed between two endpoints. Researchers
have been exploiting the flexibility of SR to route over a wide range of paths to better balance
the link utilisation when routing a large set of demands \cite{defo, hartert2015solving, steven, bhatia}.
We proposed a solution using column generation to find near optimal solutions for the TE problem \cite{CG4SR}.
A detailed explanation of our solution is given in Chapter \ref{chapter:te}.

\subsubsection*{Network monitoring}

In Chapter \ref{chapter:scmon} we explain how we
leverage segment routing to develop a network monitoring technique to detect single link failures
in a network from a single vantage point \cite{scmon}. Our technique consists of computing a set of cycles that cover
every single link of the network and to continuously send probes over those cycles in order to detect when a 
link goes down. SR is crucial to our technique as it offers the necessary routing flexibility to ensure
that we can route the probes over those cycles. 

\subsubsection*{Traffic duplication over disjoint paths}

Another application of segment routing that we explored is traffic duplication over disjoint paths.
To achieve more efficient communications for low volume traffic, we propose a solution that consists of duplicating this
traffic over several disjoint paths \cite{duplication}. Again, the ability of segment routing to forward
traffic over non shortest paths is crucial to forward traffic over disjoint paths.

We also show that we can exploit some properties of segment routing paths to provide disjoint paths that a 
robust to link failures \cite{rdp}. Chapter \ref{chapter:disjoint} provides an in depth explanation of how this
was achieved.


\section{Publications}

A lot of the work presented on this thesis goes well beyond the publications listed below. 
While writing this thesis we explored each of the problems that we worked on in more depth
finding new interesting and yet unpublished results.

\begin{itemize}
 \item François Aubry, David Lebrun, Yves Deville, Olivier Bonaventure. \emph{Traffic duplication through segmentable disjoint paths} published in IFIP Networking Conference 2015 \cite{duplication}.
 \item François Aubry, David Lebrun, Stefano Vissicchio, Minh Thanh Khong, Yves Deville, Olivier Bonaventure. \emph{SCMon: Leveraging segment routing to improve network monitoring} published in IEEE INFOCOM 2016 \cite{scmon}.
 \item François Aubry, Stefano Vissicchio,	Olivier Bonaventure, Yves Deville.  \emph{Robustly disjoint paths with segment routing} published in the International Conference on emerging Networking EXperiments and Technologies (CoNEXT) 2018 \cite{rdp}.
 \item  Mathieu Jadin, Fraçois Aubry, Pierre Schaus, Olivier Bonaventure. \emph{CG4SR: Near Optimal Traffic Engineering for Segment Routing with Column Generation} published in IEEE INFOCOM 2019 \cite{CG4SR}.
\end{itemize}


\section{Structure and style}

We did our best effort for this work be as self contained as possible while at the time trying to remain focused on what is novel. 
From a theoretical point of view this
means that we define every concept that is used basing ourselves only on very general 
mathematical concepts such as sets, ordered pairs, sequences and so on. From a practical point of view,
we provide a full Java implementation of every algorithm developed in this thesis which is available in the git
repository:

\begin{center}
\url{https://github.com/yunoac/thesis}
\end{center}

We also provide a pseudocode formalization
of the main algorithms that we propose.

Most of the data that we use is also publicly available as well as the scripts to perform
the experiments described in this thesis. The only artifacts that are not available are private ISP topologies
that we are not allowed to disclose.

We hope that these choices will make it easier for others to build upon our work and develop new theoretical results
and practical applications of segment routing.

The remainder of this thesis is organized as follows. In Chapter \ref{chapter:graphs} we provide the necessary
background in graph theory which forms the basic building blocks of everything else in this thesis. We start by
giving a graph representation of computer networks which slight differs from the traditional graph definitions.
Our model will more easily enable us to model common features of networks such as, for instance, parallel links. Then, we provided
a short explanation of the shortest path theory. Since segment routing paths are essentially concatenations of
shortest paths, understanding the properties of shortest paths is crucial  for understanding segment routing.

In Chapter \ref{chapter:dataset} we provide a description of the dataset that was used on the experimental evaluation
of our algorithm. We also analyze some of the properties related to the structure of these networks like, for instance, how 
dense they are or the path diversity between pairs of routers.

Then, in Chapter \ref{chapter:sr} we provide, to be best of our knowledge, the first theoretical study of segment routing 
that fully covers both node and adjacency segments. Here we prove some fundamental theorems which are key to
each of the applications of segment routing that we developed in this thesis. We also provide an analysis
of segment routing on the topologies of our dataset.

Chapter \ref{chapter:sr-optimal} explores the problem of computing optimal segment routing paths with respect to several different metrics.
We will show how to compute segment routing paths of minimum latency which is 
useful if one wants to use SR to route traffic between two points over a path of minimum latency. As we will see
later, this problem also arises in one form or another as a subproblem of each and every one of the applications
that we studied.
We also provide an algorithm for computing segment routing paths with maximum bandwidth.

Next, in Chapter \ref{chapter:te} we propose a solution for the traffic engineering problem with segment routing 
based on column generation.
We compare our approach with existing ones and show that our solutions are near optimal and provide a better lower bound
that traditional techniques based on the multi-commodity flow problem.

In Chapter \ref{chapter:scmon} we present a solution for detecting link failures in a network. Our solution is granular to
the point that it can detect link failures within link bundles. We will also show that using the tools developed 
in Chapter \ref{chapter:sr}, we can provide exact bounds on the minimum number of segments required in any
cycle cover of a network.

Finally, we study the problem of computing sets of disjoint paths with SR in Chapter \ref{chapter:disjoint}. We show that
traffic duplication can be used to reduce the latency of network communications. We also exploit properties of
segment routing paths to prove pairs of disjoint paths that are robust to any link failure.


\section{Experimental setting}

Apart from the result in chapter \ref{chapter:te}, all experimental were ran on a DELL Latitude E5450 computer with a
Intel Core i5-4310U at 2.00GHz and 8 GB of RAM.

The results from chapter \ref{chapter:te} were obtained by running on a computer with 32 CPUs at 2.60GHz, 128GB of RAM.
Our solution does not actually need 128GB of RAM but it is able to take advantage of the 32CPUs thanks to the multithreading.

All code is written in Java 1.8 JVM. The mixed integer programming solved that we use is Gurobi v8.0.

