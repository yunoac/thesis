\chapter{Network and routing model}
\label{chapter:graphs}

\section*{Introduction}

A natural way to provide a mathematical model of a computer network is to represent it with a graph. 
Graphs consist of a collection
of objects called \emph{nodes} and a relation between these objects called \emph{edges}. In the case of computer networks, 
the nodes correspond to the routers and the edges to the links. Graphs provide an abstraction that captures
the connectivity relationships and topological structure of its objects. It abstracts from properties
like, say, physical position of the routers. 

Such a structure can be augmented with functions on the edges in order to be able to provide a less abstract	
model and represent some properties of these connections such as the distance between two routers connected
by a network link or the bandwidth of the links connecting them.

In this chapter we describe our computer network model and prove the core properties about shortest paths.
These properties act as the building blocks for understanding segment routing.

\section{Graph theory}

\begin{definition}
A \emph{directed graph} $G$ is a pair $(V, E)$ where $V$ is a finite set and $E \subseteq V \times V$. 
The set $V$ is called \emph{nodes} of $G$ and $E$ is called the set of \emph{edges} of $G$. Whenever $G$ might
not be clear from the context, we will write $V(G)$ and $E(G)$ to represented its set of nodes and edges, respectively.
\end{definition}

\begin{definition}
A \emph{weighted directed graph} if a triple $(V, E, w)$ where $(V, E)$ is a directed graph and
$w : E \rightarrow \mathbb{R}$ is a function representing edge weights. In some situations we need
to model several different weights on the edges. In this case the weighted graph is represented by
a tuple $(V, E, w_1, \ldots, w_k)$ with each $w_i : E \rightarrow \mathbb{R}$.
\end{definition}

As mentioned above, graphs offer a good abstraction for representing a computer network because the main features that
we need to care about when designing most optimization algorithms for networking problems
consist of which routers are interconnected and the characteristics these connections. Typically each
link is characterized by three numbers: the IGP weight, the capacity (bandwidth) and the latency.

In this work we propose a slightly different model of networks. We add an index on each edge so that
we can represent parallel links without having to use multi-sets. These indexes also make it easier, both in
theory and in practice, to work with subgraphs and logical operations on these.

\begin{definition}
A \emph{network} is a tuple $G = (V, E, \igp, \bnd, \lat)$ where
$V = \{0, \ldots, n - 1\}$ for some $n \in \mathbb{N}$ and
$E \subseteq V \times V \times \mathbb{N}$ such that 
$$
\{ i \mid (u, v, i) \in E \} = \{ 0, \ldots, |E| - 1 \}.
$$

The weight functions are such that $\igp : E \rightarrow \mathbb{N}^+$
represents the IGP weights configured on the links $\bnd : E \rightarrow \mathbb{N}$ represents the links
capacities and $\lat : E \rightarrow \mathbb{R}^+$ represents the link latencies.

We refer to the set of nodes of $G$ as $V(G)$ and the set of edges as $E(G)$. We denote the number of nodes
as $|V(G)|$, the number of edges as $|E(G)|$ and define $|G| = |V(G)| + |E(G)|$.
\end{definition}

When the index is not relevant in a given context, we will simplify the notation an omit it.
For example, if we want to refer to an arbitrary edge in a network between nodes $u$ and $v$ 
we will simply write $(u, v) \in E(G)$ instead of $(u, v, i) \in E(G)$.
Sometimes it will be useful to be able to refer to all edges between two given nodes. For this,
we defined the following.

\begin{definition}
Let $G$ be a network and $u, v \in V(G)$. We denote the set of edges between $u$ and $v$ by
$$
E(G, u, v) = \{ (u, v, i) \mid  (u, v, i) \in E(G) \}.
$$
Whenever $G$ is clear from the contest we will omit it from the notation.
\end{definition}

What follows is another useful notation that allows to easily refer to the end-point of a given 
edge in the network.

\begin{definition}
Let $G$ be a network and $e = (u, v, i) \in E(G)$. We say that $e$ is an edge from $u$ to $v$
and denote $e^1 = u$, $e^2 = v$ and $\idx(e) = i$.
\end{definition}

Next we define the concept of a \emph{path} in the network. Traditionally, a path is represented by a sequence
of nodes that are traversed. So $(\node{a}, \node{b}, \node{c}, \node{d})$ would represent a path
from \node{a} to \node{d} that visits \node{b} and \node{c} in that order. However, since we can have
parallel links in a network, specifying the nodes that are visited does not fully describe the path.
Instead we represent a path by the sequence of edges that are visited.

\begin{definition}
A \emph{path} $p$ in a network $G$ is a sequence $(e_1, \ldots, e_l)$ such that for each 
$i = 1, \ldots, l$, $e_i \in E(G)$ and if $i < l$ then $e^2_i = e^1_{i + 1}$.

We define
$$
\igp(p) = \sum_{i = 1}^l \igp(e_i)
$$
and similarly for $\bnd(p)$ and $\lat(p)$.

We say that $p$ is a path from $u$ to $v$ if $e^1_1 = u$ and $e^2_l = v$.

If $p$ is a path ending at $v$, that is, if $e^2_l = v$, then we define $p \oplus (v, u) = (e_1, \ldots, e_l, (v, u))$ 
If $v$ is a node and $p$ is a path starting at $v$ we define $(v) \oplus p = p$. If $p$ ends at $v$ we define 
$p \oplus (v) = p$. This notation will be convenient later on.

The set of edges of $p$ is $E(p) = \{ e_1, \ldots, e_l \}$ and the set of nodes of
$p$ is $V(p) = \{ e^1_1, e^2_1, \ldots, e^1_l, e^2_l \}$.

We say that path $p$ has \emph{node sequence} $(e^1_1, e^2_1, \ldots, e^1_l, e^2_l)$.
A path is said to be \emph{simple} if its node sequence does not contain repeated elements.

A \emph{cycle} is a path such that $e^1_1 = e^2_l$.
\end{definition}

For most networks, whenever there is a link connecting a router $u$ to another router $v$, there usually also 
is a link connecting router $v$ to router $u$. To capture these networks we define the concept of 
symmetric network.

\begin{definition}
A network $G$ is said to be \emph{symmetric} if for every link $(u, v, i)$ there exists another link
$(v, u, i')$ with a bijective correspondence $\rev$ between links $(u, v, i)$ with $u < v$
and links $(u, v, i)$ with $u > v$.
\end{definition}

In general we do not assume that the IGP weights are the same in both directions,
that is, we can have some $e \in E(G)$ in a symmetric network such that $\igp(e) = \igp(\rev(e))$.
However, in some contexts we will need this assumption. Whenever this is the case, that is, whenever for each
$e \in E(G)$, $\igp(e) \neq \igp(\rev(e))$ we say that $\igp$ is \emph{symmetric}.

Before we proceed, we want to make a comment about the conventions used to represent networks in the figures used in this
thesis. In most examples (if not all), we assume that the network is symmetric with unit IGP weights. When this is the case, we draw a single
line between $u$ and $v$ representing both these edges. If several parallel links exists, then several such
lines will be drawn. Figure \ref{fig:links} illustrates this.

\begin{figure}
\begin{center}
\begin{tabular}{ccc}
\begin{tikzpicture}
\draw[white] (-1, 1) rectangle (3, -3.5);
\node[scale=0.15] (a) at (0, 0) {\router{a}{router}};
\node[scale=0.15] (b) at (2, 0) {\router{b}{router}};
\draw[line width=2, black] (a) edge[above, sloped, ->] node[black,font=\bfseries] {\tiny \texttt{}} (b);
\node[anchor=north] at (1, -0.5) {
  \begin{varwidth}{3.25cm}
    A network with a single link between \node{a} and \node{b}
  \end{varwidth}
};
\end{tikzpicture}
&
\begin{tikzpicture}
\draw[white] (-1, 1) rectangle (3, -3.5);
\node[scale=0.15] (a) at (0, 0) {\router{a}{router}};
\node[scale=0.15] (b) at (2, 0) {\router{b}{router}};
\draw[line width=2, black] (a) edge[above, sloped] node[black,font=\bfseries] {\tiny \texttt{}} (b);
\node[anchor=north] at (1, -0.5) {
  \begin{varwidth}{3.25cm}
   A network with a bidirectional link between \node{a} and \node{b}. This means that it has two edges:
   $(\node{a}, \node{b}, 1)$ and $(\node{b}, \node{a}, 2)$
  \end{varwidth}
};
\end{tikzpicture}
&
\begin{tikzpicture}
\draw[white] (-1, 1) rectangle (3, -3.5);
\node[scale=0.15] (a) at (0, 0) {\router{a}{router}};
\node[scale=0.15] (b) at (2, 0) {\router{b}{router}};
\draw[line width=2, black] (a) edge[above, sloped, bend left] node[black,font=\bfseries] {\tiny \texttt{}} (b);
\draw[line width=2, black] (a) edge[above, sloped, bend right] node[black,font=\bfseries] {\tiny \texttt{}} (b);
\node[anchor=north] at (1, -0.5) {
  \begin{varwidth}{3.25cm}
    A network with two bidirectional links between \node{a} and \node{b}. This means that it has four edges:
    $(\node{a}, \node{b}, 1)$, $(\node{b}, \node{a}, 2)$,  $(\node{a}, \node{b}, 3)$ and $(\node{b}, \node{a}, 4)$
  \end{varwidth}
};
\end{tikzpicture}
\end{tabular}
\\
\end{center}
\caption{In and out neighbors of a node.}
\label{fig:links}
\end{figure}

To represent a specific link or a path in the network, we draw colored directed edges on top the 
lines representing the link. 

By default, numbers above links represent IGP weights. 
As we said, in most examples we use unit IGP weights on every link, meaning that shortest paths are measured
in terms of the number of links on the path. In this case we omit the weight to lighten the figures. 
Whenever this is not the case, the numbers on top of the
links represent the IGP weight. If we want to represent some other link metric, it will be explicitly stated
in the example what the numbers represent. This closes the parenthesis about graphical representations
of networks in this work. \\

The following definitions make it easy to refer to the neighbors of a given node. We define the set of 
edges going into and out of a node as well as the sets of neighbors consisting of their endpoints.

\begin{definition}
Let $G$ be a network and $v \in V(G)$. We define
\begin{enumerate}[i)]
 \item $\oute(G, v) = \{ (v, u, i) \mid (v, u, i) \in E(G) \}$
 \item $\ine(G, v) = \{ (u, v, i) \mid (u, v, i) \in E(G) \}$
 \item $\outn(G, v) = \{ u \mid (v, u, i)  \in E(G) \}$
 \item $\inn(G, v) = \{ u \mid (u, v, i) \in E(G) \}$
\end{enumerate}
\end{definition}

When the underlying graph $G$ is clear from context, we omit it to lighten the notation.
Figure \ref{fig:inout} illustrates this definition.


\begin{figure}
\begin{center}
\begin{tikzpicture}
\def\x{0}
\def\y{0}

\node[scale=0.15] (a) at (0.5 + \x,  0.5 + \y) {\router{a}{green}};
\node[scale=0.15] (b) at (0.5 + \x, -1.0 + \y) {\router{b}{green}};
\node[scale=0.15] (c) at (2.5 + \x,  0.0 + \y) {\router{c}{router}};
\node[scale=0.15] (d) at (4.5 + \x,  0.0 + \y) {\router{d}{orange!20!white}};
\node[scale=0.15] (e) at (4.5 + \x, -1.5 + \y) {\router{e}{orange!20!white}};
\node[scale=0.15] (f) at (2.5 + \x,  -2.5 + \y) {\router{f}{green}};

\draw[line width=2, black] (a) edge[above, sloped] node[black,font=\bfseries] {\tiny \texttt{}} (b);
\draw[line width=2, black] (b) edge[sloped] node[black,font=\bfseries] (bc) {\tiny \texttt{}} (c);
\draw[line width=2, black] (e) edge[sloped] node[black,font=\bfseries] (ec) {\tiny \texttt{}} (c);
\draw[line width=2, black] (b) edge[above, sloped] node[black,font=\bfseries] {\tiny \texttt{}} (e);
\draw[line width=2, black]  (d) edge[above, sloped] node[black,font=\bfseries] {\tiny \texttt{}} (e);
\draw[line width=2, black]  (c) edge[sloped] node[black,font=\bfseries] (cd) {\tiny \texttt{}} (d);
\draw[line width=2, black]  (a) edge[sloped] node[black,font=\bfseries] (ac) {\tiny \texttt{}} (c);
\draw[line width=2, black]  (f) edge[sloped] node[black,font=\bfseries] {\tiny \texttt{}} (b);
\draw[line width=2, black]  (f) edge[sloped] node[black,font=\bfseries] {\tiny \texttt{}} (e);

\draw[line width=3, darkgreen] (a) edge[->] (c);
\draw[line width=3, darkgreen] (b) edge[->] (c);

\draw[line width=3, orange] (c) edge[->] (d);
\draw[line width=3, orange] (c) edge[->] (e);

\node (d-) at (1.5, 1.5) {$\ine(\node{c})$};
\node (d+) at (3.5, 1.5) {$\oute(\node{c})$};

\draw[->] (d-) -- (ac.east);
\draw[->] (d-) -- (bc.west);

\draw[->] (d+) -- (ec.west);
\draw[->] (d+) -- (cd.east);

\draw[dashed] (0.5, -0.25) ellipse (0.7 and 1.7);
\draw[dashed] (4.5, -0.75) ellipse (0.7 and 1.7);

\node at (-0.75, -0.25) {$\inn(\node{c})$};
\node at (4.5+1.25, -0.75) {$\outn(\node{c})$};


\end{tikzpicture}
\end{center}
\caption{In and out neighbors of a node.}
\label{fig:inout}
\end{figure}

We conclude this section with a definition of subnetwork.

\begin{definition}
Let $G, H$ be two networks. If $V(G) = V(H)$ and $E(H) \subseteq E(G)$ we say that
$H$ is a \emph{subnetwork} of $G$ and write $H \subseteq G$. The weights $\igp, \lat$
and $\bnd$ are defined on $H$ by restricting them to $E(H)$. 
\end{definition}

Traditionally subgraphs are defined in a way that also restricts the set of nodes in the result.
More precisely, the set of nodes is often restricted to the set of nodes that are the endpoints
of some edge in the subgraph. We find it more convenient to keep the same set of nodes as 
this is more in line with how the algorithms proposed in this thesis are implemented.

\section{Shortest paths}

As mentioned in the introduction, segment routing leverages shortest path routing in order to
forward packets over the network. For this reason, understanding shortest paths is central to 
understanding segment routing. In this section we provide an overview of the most important
properties shortest paths and of the existing algorithms used to compute such paths.

Unless stated otherwise, whenever we refer to a shortest path in a given network $G$, we mean
a shortest path with respect to the $\igp$ weight function.

\begin{definition}
Let $G$ be a network and $s, t \in V(G)$. 
A shortest path between $s$ and $t$ is a path $p$ from $s$ to $t$ such that
$\igp(p)$ is minimum.
\end{definition}

Shortest paths are not necessarily unique. When several shortest paths exist between two nodes we say
that there is Equal-cost multi-path (ECMP) between those nodes. In order to characterize the set of network
links where packets might pass when forwarded from a given network router, we  need to be able to
refer to the subgraph containing all shortest paths starting from a given node in the network. This 
concept is formalized with the following definition.

\begin{definition}
Let $G$ be a network and $r \in V(G)$. We define the \emph{shortest path subnetwork rooted at $r$} as
$\sp(G, r) = (V, E')$ where $e \in E'$ if there exists $v \in V(G)$ and a shortest path $p$ from
$r$ to $v$ such that $e \in E(p)$. Whenever $G$ is clear from the context, we  omit it from 
the notation and simply write $\sp(r)$.
\end{definition}

\begin{definition}
Let $G$ be a network. We say that $G$ is \emph{acyclic} if it contains no cycles.
\end{definition}

It is not hard to see that the set of all shortest path between starting from a given node is
acyclic.

\begin{lemma}
\label{lemma:dag}
Let $G$ be a network and $r \in V(G)$. Then $\sp(r)$ is an acyclic subnetwork of $G$.
\end{lemma}

\begin{proof}
Suppose that there exists a cycle $c = (e_1, \ldots, e_l)$ on $\sp(r)$.
Then, by definition
there exist shortest paths $p_1, \ldots, p_l$ such that $p_i$ is a shortest path starting at $r$ containing $e_i$.
For each $i$, write $e_i = (u_i, v_i)$, $w_i = \igp(p_1) - \igp(e_i)$ and $j_i$ such that

\[ j_i =
  \begin{cases}
    i + 1     & \quad \text{if } i + 1 < l \\
    1  & \quad \text{if } i = l
  \end{cases}
\]

Before we proceed with the general case, let's see the proof in an example with three nodes. 
So suppose that $c = (e_1, e_2, e_3) = ((a, b), (b, c), (c, d))$ as shown in
Figure \ref{fig:dag_intuition}. By definition, $w_1 + \igp((a, b))$ is the cost of a shortest path from $r$ to $b$. Since
$w_2$ is also the cost of a path from $r$ to $b$ we have that $w_1 + \igp((a, b)) \leq w_2$. Since $\igp((a, b)) > 0$
we conclude that $w_1 < w_2$. In the same way we can see that $w_2 < w_3$ and $w_3 < w_1$ leading to
the contradiction that $w_1 < w_1$.

\begin{figure}[H]
\begin{center}
\begin{tikzpicture}
\node[scale=0.15] (r) at (0, 0) {\router{r}{lightgray}};
\node[scale=0.15] (a) at (3, -1) {\router{a}{lightgray}};
\node[scale=0.15] (b) at (3, 1) {\router{b}{lightgray}};
\node[scale=0.15] (c) at (5, 0) {\router{c}{lightgray}};
\draw[line width=2] (a) edge[above, sloped, ->] node[black,font=\bfseries] {\footnotesize $e_1$} (b);
\draw[line width=2] (b) edge[above, sloped, ->] node[black,font=\bfseries] {\footnotesize $e_2$} (c);
\draw[line width=2] (c) edge[above, sloped, ->] node[black,font=\bfseries] {\footnotesize $e_3$} (a);
\path (r) edge[very thick, out=0, in=180, wavy, ->] node[black,font=\bfseries, above right] {$w_1$} (a);
\path (r) edge[very thick, out=30, in=130, wavy, ->] node[black,font=\bfseries, above left] {$w_2$} (b);
\path (r) edge[very thick, out=270, in=240, wavy, ->] node[black,font=\bfseries, above] {$w_3$} (c);
\end{tikzpicture}
\end{center}
\caption{Proof intuition on a cycle with $3$ edges. Wiggly arrows represent shortest path.}
\label{fig:dag_intuition}
\end{figure}

In the general case, we have that $w_i + \igp(e_i) \leq w_j$ because $w_i + \igp(e_i)$ is by definition the cost
of the shortest path from $r$ to $v_i$ and $w_j$ is the cost of a path from $r$ to $v_i$. Since for each $i$,
$\igp(w_i) > 0$ we conclude that $w_i < w_j$. As this is true for all $i$ we have
$w_1 < w_2 < \ldots < w_l < w_1$ which is impossible. Therefore such a cycle cannot exist.
\end{proof}

\begin{definition}
Given a network $G$ we denote the shortest path distance (minimum IGP cost) between two nodes by $\dist(G, u, v)$ or
simply $d(x, y)$ if $G$ is clear from the context. 
\end{definition}

Note that with the above definition, $p$ is a shortest path from $u$ to $v$ if and only if $\igp(p) = \dist(u, v)$.

One important property of shortest paths that we will use often when proving properties of segment routing is that
a sub-path of a shortest path is also a shortest path.

\begin{proposition}
\label{prop:subsp}
Let $G$ be a network and $p = (e_1, \ldots, e_l)$ a shortest path on $G$. Then for each
$i, j \in \{1, \ldots, l\}$ with $i \leq j$, $(e_i, \ldots, e_j)$ a shortest path.
\end{proposition}

\begin{proof}
Let $G$ be a network and $p = (e_1, \ldots, e_l)$ a shortest path on $G$ and $i, j \in \{1, \ldots, l\}$
with $i \leq j$.
Let $v = e^1_i$ and $u = e^2_j$. Suppose that $(e_i, \ldots, e_j)$ is not a shortest path. 
There there exists another shorter path $q = (f_1, \ldots, f_r)$ from $u$ to $v$. 
Consider $p' = (e_1, \ldots, e_{i - 1}, f_1, \ldots, f_r, e_{j + 1}, \ldots, e_l)$.
Since $q$ is a path from $u$ to $v$ and since $u = f^1_i = e^1_i = e^2_{i - 1}$
and $v = f^2_r = e^2_{j} = e^1_{j + 1}$ , we have that $p'$ is also a 
path from $u$ to $v$. This path is shorter than $p$ since
\begin{align*}
\igp(p') & = \sum_{k = 1}^{i - 1} \igp(e_k) + \sum_{k = 1}^r \igp(f_k) + \sum_{k = j + 1}^l \igp(e_i) \just{definition of $\igp(p')$} \\
         & =  \sum_{k = 1}^{i - 1} \igp(e_k) + \igp(q) + \sum_{k = j + 1}^l \igp(e_i) \just{definition of $\igp(q)$}  \\
         & <  \sum_{k = 1}^{i - 1} \igp(e_k) + \sum_{k = i}^j \igp(e_k) + \sum_{k = j + 1}^l \igp(e_i)  \just{$q$ is shorter than $(e_i, \ldots, e_j)$} \\
         & = \sum_{k = 1}^l \igp(e_k) = \igp(p) \just{definition of $\igp(p)$}
\end{align*}
This contradicts the fact that $p$ is a shortest path.
\end{proof}

The shortest path subgraphs are easily characterized by the values of $\dist(u, v)$ as shown in the following proposition.

\begin{proposition}
\label{prop:sp-properties}
Let $G$ be a network $v, u \in V(G)$ and $e \in E(G)$. Then
\begin{enumerate}[i)]
 \item There exists a shortest path containing $e$ if and only if $\dist(e^1, e^2) = \igp(e)$.
 \item Edge $e$ belongs to a shortest path from $v$ to $e^2$ if and only if 
 $\dist(v, e^2) = \dist(v, e^1) + \igp(e)$. 
 \item There exists a shortest path from $v$ to $u$ containing edge $e$ if and only if
 $\dist(v, e^1) + \igp(e) + \dist(e^2, u) = \dist(v, u)$.
\end{enumerate}
\end{proposition}

\begin{proof}
Let $G$ be a network $v, u \in V(G)$ and $e \in E(G)$.
\begin{enumerate}[i)]
 \item $(\Rightarrow)$ Suppose that there exists a shortest path $p = (e_1, \ldots, e_l)$ containing edge $e$.
 Then $e = e_i$ for some $i$ and by Proposition \ref{prop:subsp} the sub-path $(e_i)$ of $p$ is a shortest path.
 Thus $\dist(e^1, e^2) = \igp(e)$. 
 
 $(\Leftarrow)$ If $\dist(e^1, e^2) = \igp(e)$ then by definition $p = (e)$ is a shortest path.
 
 \item $(\Rightarrow)$ Suppose that $e$ belongs to a shortest path $p = (e_1, \ldots, e_l)$ from $v$ to $y$.
 Since $p$ is a shortest path we must have $e_l = e$ or otherwise, since IGP weights are positive, $p$ would contain a positive cycle. 
 Therefore
 $$
 \dist(v, e^2) = \sum_{i = 1}^l \igp(e_i) = \sum_{i = 1}^{l - 1} \igp(e_i) + \igp(e)
 $$
 By Proposition \ref{prop:subsp}, $(e_1, \ldots, e_{l - 1})$ is a shortest path between 
 $v$ and $e^1$ and thus
 $$
 \sum_{i = 1}^{l - 1} \igp(e_i) = \dist(v, e^1)
 $$
 showing that $\dist(v, e^2) = \dist(v, e^1) + \igp(e)$.
 
 $(\Leftarrow)$ Suppose that $\dist(v, e^2) = \dist(v, e^1) + \igp(e)$ and let $p = (e_1, \ldots, e_l)$
 be a shortest path from $v$ to $e^1$. Then $p \oplus e = (e_1, \ldots, e_l, e)$ has cost 
 $\igp(p) = \dist(v, e^1) + \igp(e) = \dist(v, e^2)$ showing that $p \oplus e$ is a shortest path between $e^1$ and $e^2$.
 
 \item $(\Rightarrow)$ Suppose that there exists a shortest path $p = (e_1, \ldots, e_l)$ from $v$ to $u$ containing edge $e$.
 Assume that $e = e_i$ for some $i \in \{1, \ldots, l\}$. Then
 \begin{align*}
 \dist(v, u) & = \igp(p) \just{$p$ is shortest path from $v$ to $u$} \\
         & = \igp((e_1, \ldots, e_{i - 1})) + \igp(e_i) + \igp((e_{i + 1}, \ldots, e_l)) \just{def of $\igp(p)$} \\
         & = \igp((e_1, \ldots, e_{i - 1})) + \igp(e) + \igp((e_{i + 1}, \ldots, e_l)) \just{$e_i = e$}  \\
         & = \dist(v, e^1) + igp(e) + \dist(e^2, u) \just{sub-paths are shortest paths}
 \end{align*}

 $(\Leftarrow)$ Suppose that $\dist(v, e^1) + \igp(e) + \dist(e^2, u) = \dist(v, u)$. Let $p$ be a shortest
 path from $v$ to $e^2$ and $q$ be a shortest path from $e^2$ to $u$. Then $p \oplus e \oplus q$ is a path from
 $v$ to $u$ containing edge $e$ of cost $\dist(v, e^1) + \igp(e) + \dist(e^2, u)$. Since $\dist(v, e^1) + \igp(e) + \dist(e^2, u) = \dist(v, u)$
 it follows that $p \oplus e \oplus q$ is a shortest path from $v$ to $u$.
 
\end{enumerate}
 
\end{proof}

Next, we are going to prove some properties about shortest paths on symmetric networks.

\begin{definition}
Let $G$ be a symmetric network and $X \subseteq(E(G))$. We define
$$
\rev(X) = \{ \rev(e) \mid e \in X \}.
$$
\end{definition}

\begin{definition}
Let $G$ be a symmetric network and $p = (e_1, \ldots, e_l)$ a path on $G$. We define
$$
\rev(p) = ( \rev(e_l), \ldots, \rev(e_1) ).
$$
\end{definition}

\begin{lemma}
\label{lemma:rev-path}
Let $G$ be a symmetric network $u, v \in V(G)$ and $p = (e_1, \ldots, e_l)$ be a path on $G$ from $u$ to $v$.
If $\igp$ is symmetric then $\rev(p)$ is a path from $v$ to $u$ with $\igp(p) = \igp(\rev(p))$.
\end{lemma}

\begin{proof}
Let $i \in \{l, \ldots, 2\}$. Since $p$ is a path, we have $e^2_{i - 1} = e^1_{i}$. Thus
$\rev(e_i)^2 = e^1_i = e^2_{i - 1}  = \rev(e_{i - 1})^1$ so that $\rev(p)$ is a path.
It starts at $\rev(e_l)^1 = e^2_l = v$ and ends at $\rev(e_1)^2 = e^1_1 = u$. Finally,
by definition, we have that $\igp(e_i) = \igp(\rev(e_i))$ for each $i$ so $\igp(p) = \igp(\rev(p))$.
\end{proof}

\begin{definition}
Let $G$ be  a symmetric network and $H \subseteq G$. We define $\rev(H) = (V, \rev(E(H)))$.
\end{definition}

\begin{lemma}
\label{lemma:sp-reverse}
Let $G$ be a symmetric network, $u, v \in V(G)$ and $p$ a shortest path from $u$ to $v$.
If $\igp$ is symmetric then $\rev(p)$ is a shortest path from $v$ to $u$.
\end{lemma}

\begin{proof}
We already from Lemma \ref{lemma:rev-path} that $\rev(p)$ is a path from $v$ to $u$ with $\igp(p) = \igp(\rev(p))$.
Suppose that it is not a shortest path. Then there exists a path $q$ from $v$ to $u$ such that $\igp(q) < \igp(\rev(p))$. Therefore,
$$
\igp(\rev(q)) = \igp(q) < \igp(\rev(p)) = \igp(p)
$$
contradicting the fact that $p$ is a shortest path from $u$ to $v$ since $\rev(q)$ is also a path from $u$ to $v$.
\end{proof}

\begin{corollary}
\label{cor:dag-rev}
Let $G$ be a symmetric network and $u, v \in V(G)$. If $\igp$ is symmetric then $\sp(u, v) = \rev(\sp(v, u))$. 
\end{corollary}

\begin{proof}
By definition, $V(\sp(u, v)) = V(\rev(\sp(v, u)))$. Hence we need to show that $E(\sp(u, v)) = E(\rev(\sp(v, u)))$. 
Let $e \in E(\sp(u, v))$. By definition of $\sp(u, v)$, there exists a shortest path $p$ from $u$ to $v$. By Lemma
\ref{lemma:sp-reverse} $\rev(p)$ is a shortest path from $v$ to $u$ that contains $e$. Hence $E(\rev(p)) \subseteq \sp(v, u)$. 
By definition of $\rev(p)$, there must be $e' \in \rev(p)$ such that $e = \rev(e')$. Therefore, since $e' \in \sp(v, u)$,
$e = \rev(e') \in \rev(\sp(v, u))$. We conclude that $E(\sp(u, v)) \subseteq E(\rev(\sp(v, u)))$. The proof that 
$E(\sp(u, v)) \supseteq E(\rev(\sp(v, u)))$ is analogous.
\end{proof}

\section*{Conclusion}

In this chapter we have defined the graph definition that we use throughout this thesis and proved the basic
properties of shortest path adapted to our definitions. These properties will be fundamental in the remainder of this
thesis as we build upon them to develop our segment routing theory and prove our results.
