\section{Minimal segmentations}

In this section we study the problem of efficiently representing paths in the network with SR.
Suppose that you have a path in your network that you want to use to send traffic between two nodes. Traditional
shortest path routing does not allow you to forward traffic on it unless that path already is a shortest paht with
respect to the $\igp$ weights configured on the links. SR makes it possible, at least in theory, to forward traffic
on \emph{any} path in the network since we can always just add all edges of the path as adjacency segments
in the segment routing stack. In practice however, this solution cannot be implemented because, as we mentioned above, 
due to hardware limitations of the routers currently deployed in the networks, 
the stack size is often limited to a few segments \cite{Tantsura_SID:2017}.

This motivates the problem of, given a path $p$ in the network, finding a sr-path $\sr{p}$ that represents $p$. That is, a sr-path $\sr{p}$
such that if traffic is routed over it, packets will traverse exactly the edges of $p$ in order. To do that, we need a formal definition of what 
does it mean for a sr-path to represent a path in the network.

\begin{definition}
Let $\sr{p} = \langle x_1, \ldots, x_l \rangle$ be a deterministic sr-path. We define
$$
\seq(\sr{p}) = x_1 \oplus \sp(x^2_1, x^1_2) \oplus x_2 \oplus \ldots \oplus \sp(x^2_{l - 1}, x^1_l) \oplus x_l
$$
\end{definition}

Note that since $\sr{p}$ is deterministic, $\seq(\sr{p})$ corresponds to a path in $G$ since each $\sp(x^2_{i - 1}, x^1_i)$
is a single path.

\begin{definition}
Let $G$ be a network and $p$ a path on $G$. A deterministic sr-path $\sr{p} = \langle x_1, \ldots, x_l \rangle$ is said to be a segmentation of $p$ if and only if
$p = \seq(\sr{p})$.
\end{definition}

Note that we restrict ourselves to deterministic sr-paths since the equality in the above definition would never hold if
for some $i$, $\sp(x^2_i, x^1_{i + 1})$ was not a path. Moreover, $\oplus$ is only defined if both operands are paths.
We start by showing that any path possess a segmentation.

\begin{lemma}
\label{lemma:seg_exists}
Let $G$ be a network and $p$ a path on $G$. Then, there exists a sr-path $\sr{p}$ such that $\sr{p}$ is a segmentation of
$p$. Moreover, any path admits a segmentation of cost at most $2|E(G)|$.
\end{lemma}

\begin{proof}
Let $p = (e_1, \ldots, e_l)$ be a path on $G$ and let $\sr{p} = \langle e_1, \ldots, e_l \rangle$. For each $i = 1, \ldots, l - 1$
we have that $\sp(e^2_i, e^1_{i + 1}) = \emptyset$ since $e^2_i = e^1_{i + 1}$. Thus
\begin{align*}
e_1 \oplus \sp(e^2_1, e^1_2) \oplus e_2 \oplus \ldots \oplus \sp(e^2_{l - 1}, e^1_l) \oplus e_l & = e_1 \oplus e_2 \oplus \ldots \oplus e_l = p.
\end{align*}
\end{proof}

Note that previous models of segment routing considered only node segments and therefore this result was not true. In such models it is impossible
to represent some network paths in terms of SR. As we saw in Chapter \ref{chapter:dataset}, there are about $1\%$ of the network links
that do not belong to a shortest path and $24\%$ of pairs of nodes with ECMP. We mentioned that those were the conditions that might lead to the
need of using an adjacency segment. We characterize in Proposition \ref{prop:adj-segs} exactly which situations lead to the need of using
an adjacency segment.

We call the segmentation used in the proof of Lemma \ref{lemma:seg_exists} the \emph{edge segmentation} of $p$.
Consider the path $((\node{a}, \node{b}), (\node{b}, \node{c}), (\node{c}, \node{d}), (\node{d}, \node{g}), (\node{g}, \node{i}))$ shown in green in Figure \ref{fig:min-seg-example}.
We already saw in Lemma \ref{lemma:seg_exists} above that $\sr{p} = \langle (\node{a}, \node{b}), (\node{b}, \node{c}), (\node{c}, \node{d}), (\node{d}, \node{g}), (\node{g}, \node{i}) \rangle$
is a segmentation of $p$. But this is not the unique segmentation of $p$. For example $\sr{q} = \langle \node{a}, (\node{b}, \node{c}), \node{g}, \node{i} \rangle$ is
also a segmentation of $p$. We have
\begin{align*}
e_1 \oplus \sp(e^2_1, e^1_2) \oplus e_2 \oplus \ldots \oplus \sp(e^2_{l - 1}, e^1_l) \oplus e_l & = \\
(\node{a}) \oplus \sp(\node{a}, \node{b}) \oplus (\node{b}, \node{c}) \oplus  \sp(\node{c}, \node{g}) \oplus (g) \oplus \sp(\node{g}, \node{i}) \oplus (i) & = \\
(\node{a}) \oplus ((\node{a}, \node{b})) \oplus ((\node{b}, \node{c})) \oplus  ((\node{c}, \node{d}), (\node{d}, \node{g})) \oplus (g) \oplus ((\node{g}, \node{i})) \oplus (i) & = \\
((\node{a}, \node{b}), (\node{b}, \node{c}), (\node{c}, \node{d}), (\node{d}, \node{g}), (\node{g}, \node{i})).
\end{align*}

\begin{figure}[H]
\begin{center}
\begin{tikzpicture}[scale=1.4]
\def\x{0}
\def\y{0}
\node[scale=0.15] (a) at (0.5 + \x,  0.5 + \y) {\router{a}{marked}};
\node[scale=0.15] (b) at (0.5 + \x, -1.0 + \y) {\router{b}{router}};
\node[scale=0.15] (c) at (2.5 + \x,  0.0 + \y) {\router{c}{router}};
\node[scale=0.15] (d) at (4.5 + \x,  0.0 + \y) {\router{d}{router}};
\node[scale=0.15] (e) at (4.0 + \x, -2.0 + \y) {\router{e}{router}};
\node[scale=0.15] (g) at (6.0 + \x,  0.5 + \y) {\router{g}{marked}};
\node[scale=0.15] (i) at (8.0 + \x,  0.0 + \y) {\router{i}{marked}};
\node[scale=0.15] (h) at (7.0 + \x, -1.5 + \y) {\router{h}{router}};
\draw[line width=2] (a) edge[above, sloped] node[black,font=\bfseries] {\tiny \texttt{}} (b);
\draw[line width=2] (b) edge[above, sloped] node[black,font=\bfseries] {\tiny \texttt{}} (c);
\draw[line width=2] (e) edge[above, sloped] node[black,font=\bfseries] {\tiny \texttt{}} (c);
\draw[line width=2] (b) edge[above, sloped] node[black,font=\bfseries] {\tiny \texttt{}} (e);
\draw[line width=2] (g) edge[above, sloped] node[black,font=\bfseries] {\tiny \texttt{}} (i);
\draw[line width=2] (i) edge[above, sloped] node[black,font=\bfseries] {\tiny \texttt{}} (h);
\draw[line width=2]  (d) edge[above, sloped] node[black,font=\bfseries] {\tiny \texttt{}} (g);
\draw[line width=2]  (d) edge[above, sloped] node[black,font=\bfseries] {\tiny \texttt{}} (e);
\draw[line width=2]  (e) edge[above, sloped] node[black,font=\bfseries] {\tiny \texttt{}} (h);
\draw[line width=2]  (g) edge[above, sloped] node[black,font=\bfseries] {\tiny \texttt{}} (h);
\draw[line width=2]  (c) edge[above, sloped] node[black,font=\bfseries] {\tiny \texttt{}} (d);
\draw[line width=2]  (a) edge[above, sloped] node[black,font=\bfseries] {\tiny \texttt{}} (b);
\draw[line width=2]  (a) edge[above, sloped] node[black,font=\bfseries] {\tiny \texttt{}} (c);

%%%%
\draw (a) edge[line width=4, dotted, darkgreen, ->] node[line width = 1, draw, solid, black, fill=green] {\footnotesize $x_1$} (b);
\draw (b) edge[line width=4, dotted, darkgreen, ->] node[line width = 1, draw, solid, black, fill=green] {\footnotesize $x_2$} (c);
\draw (c) edge[line width=4, dotted, darkgreen, ->] node[line width = 1, draw, solid, black, fill=green] {\footnotesize $x_3$} (d);
\draw (d) edge[line width=4, dotted, darkgreen, ->] node[line width = 1, draw, solid, black, fill=green] {\footnotesize $x_4$} (g);
\draw (g) edge[line width=4, dotted, darkgreen, ->] node[line width = 1, draw, solid, black, fill=green] {\footnotesize $x_5$} (i);

\draw (a) edge[line width=3, orange, above, bend right=25, ->] (b);
\draw (b) edge[line width=4, dotted, orange, bend right=30, ->] node[line width = 1, draw, solid, black, fill=orange!20!white] {\footnotesize $y_2$} (c);
\draw (c) edge[line width=3, orange, above, bend right=25, ->] (d);
\draw (d) edge[line width=3, orange, above, bend right=25, ->] (g);
\draw (g) edge[line width=3, orange, above, bend right=25, ->] (i);

\node[left = 0.1cm of a, draw, fill=orange!20!white] {\footnotesize $y_1$};
\node[above = 0.1cm of g, draw, fill=orange!20!white] {\footnotesize $y_3$};
\node[above = 0.1cm of i, draw, fill=orange!20!white] {\footnotesize $y_4$};

%%%%

\end{tikzpicture}
\end{center}
\caption{Illustration of two different segmentations of the same path $p = (\edge{a}{b}, \edge{b}{c}, \edge{c}{d}, \edge{d}{g}, \edge{g}{i})$. In green
we represent the edge segmentation $\langle \edge{a}{b}, \edge{b}{c}, \edge{c}{d}, \edge{d}{g}, \edge{g}{i} \rangle$ and in orange the segmentation
$\langle \node{a}, \edge{b}{c}, \node{g}, \node{i} \rangle$.}
\label{fig:min-seg-example}
\end{figure}

\begin{lemma}
\label{lemma:edgeseg}
Let $G$ be a network. Any path in $G$ has a segmentation with segment cost at most $2 \cdot |E(G)|$. 
\end{lemma}

\begin{proof}
Given a path $p = (e_1, \ldots, e_l)$ in $G$, we have that $\sr{p} = \langle e_1, \ldots, e_2 \rangle$ is a segmentation of
$p$ with cost $2 \cdot |E(G)|$.
\end{proof}


This shows that not all segmentations are equal. Even though they represent the same path in the network,
they do not have the same segment cost. In practice we would like the one with the minimum cost. This motivates
the following problem.

\begin{problem}{Minimum path segmentation}
\label{prob:min-seg}
Given a network $G$ and a path $p$ in $G$, find a segment $\sr{p}$ of $p$ such that $\cost(\sr{p})$ is minimal.
\end{problem}

We propose a greedy polynomial time algorithm solving Problem \ref{prob:min-seg}. The idea behind the algorithm is that we start walking over the input path and incrementally build the segmentation so that
at each step the current sr-path is a segmentation of the prefix of the path that we traversed so far. At each node, we check whether 
replacing the last node of the current segmentation by that node yields a correct segmentation. 
If it does, then we replace it move on to the next node. If it does not, then we need to either append that node to
the segmentation or the adjacency segment corresponding to the last traversed edge.

To illustrate this, consider the graph on Figure \ref{fig:min-seg-example} and
suppose that the input path is $p = ((\node{a}, \node{b}), (\node{b}, \node{c}), (\node{c}, \node{d}), (\node{d}, \node{g}), (\node{g}, \node{i}))$, shown in blue. 
The idea is to start with a sr-path $\langle \node{a} \rangle$
and follow path $p$ adding segments whenever necessary. When we arrive at node $b$ we have that $\sp(\node{a}, \node{b}) = (\node{a}, \node{b})$ so appending
$\node{b}$ to the segmentation make the paths match. Then we go to node $\node{c}$ and try to see whether $\node{c}$ can replace the last segment, $\node{b}$. 
This would give the sr-path $\langle \node{a}, \node{c} \rangle$ which is not ok since $\sp(\node{a}, \node{c}) = (a, c)$ which is not a prefix of path $p$. Thus, 
we cannot replace $\node{b}$ so we try to add $\node{c}$ getting the path $\langle \node{a}, \node{b}, \node{c} \rangle$. 
Now we are good because the current sr-path represents $(\node{a}, \node{b}, \node{c})$ which is a prefix of $p$. Then, we do the same with $\node{d}$.
Replacing $\node{c}$ by $\node{d}$ gives
the sr-path $\langle \node{a}, \node{b}, \node{d} \rangle$. This path is not ok because there are two shortest paths between nodes $\node{b}$ and $\node{d}$, namely,
$(\node{b}, \node{c}, \node{d})$ and $(\node{b}, \node{e}, \node{d})$ so $\sr{p}$ would not be deterministic. This means that, as before, we need to add $\node{d}$ to the end of the list getting
the sr-path $\langle \node{a}, \node{b}, \node{c}, \node{d} \rangle$. Next we go to $\node{g}$ and this time replacing node $\node{d}$ works because 
the unique shortest path between $\node{c}$ and $\node{g}$ is $((\node{c}, \node{d}), (\node{d}, \node{g}))$ which matches $p$. So far 
$\sr{p} = \langle \node{a}, \node{b}, \node{c}, \node{g} \rangle$ which represents the prefix  
$((\node{a}, \node{b}), (\node{b}, \node{c}), (\node{c}, \node{d}), (\node{d}, \node{g}))$ of $p$.
Finally, we process node $\node{i}$. Replacing $\node{g}$ by $\node{i}$ gives the sr-path $\langle \node{a}, \node{b}, \node{c}, \node{i} \rangle$ which does not match path $p$ because
there are two shortest paths between nodes $\node{c}$ and $\node{i}$. Thus we need to append node $\node{i}$ instead and the final sr-path
is $\sr{p} = \langle \node{a}, \node{b}, \node{c}, \node{g}, \node{i} \rangle$.

\begin{figure}[H]
\begin{center}
\begin{tikzpicture}
\def\x{0}
\def\y{0}
\node[scale=0.15] (a) at (0.5 + \x,  0.5 + \y) {\router{a}{marked}};
\node[scale=0.15] (b) at (0.5 + \x, -1.0 + \y) {\router{b}{marked}};
\node[scale=0.15] (c) at (2.5 + \x,  0.0 + \y) {\router{c}{marked}};
\node[scale=0.15] (d) at (4.5 + \x,  0.0 + \y) {\router{d}{router}};
\node[scale=0.15] (e) at (4.0 + \x, -2.0 + \y) {\router{e}{router}};
\node[scale=0.15] (g) at (6.0 + \x,  0.5 + \y) {\router{g}{marked}};
\node[scale=0.15] (i) at (8.0 + \x,  0.0 + \y) {\router{i}{marked}};
\node[scale=0.15] (h) at (7.0 + \x, -1.5 + \y) {\router{h}{router}};
\draw[line width=2] (a) edge[above, sloped] node[black,font=\bfseries] {\tiny \texttt{}} (b);
\draw[line width=2] (b) edge[above, sloped] node[black,font=\bfseries] {\tiny \texttt{}} (c);
\draw[line width=2] (e) edge[above, sloped] node[black,font=\bfseries] {\tiny \texttt{}} (c);
\draw[line width=2] (b) edge[above, sloped] node[black,font=\bfseries] {\tiny \texttt{}} (e);
\draw[line width=2] (g) edge[above, sloped] node[black,font=\bfseries] {\tiny \texttt{}} (i);
\draw[line width=2] (i) edge[above, sloped] node[black,font=\bfseries] {\tiny \texttt{}} (h);
\draw[line width=2]  (d) edge[above, sloped] node[black,font=\bfseries] {\tiny \texttt{}} (g);
\draw[line width=2]  (d) edge[above, sloped] node[black,font=\bfseries] {\tiny \texttt{}} (e);
\draw[line width=2]  (e) edge[above, sloped] node[black,font=\bfseries] {\tiny \texttt{}} (h);
\draw[line width=2]  (g) edge[above, sloped] node[black,font=\bfseries] {\tiny \texttt{}} (h);
\draw[line width=2]  (c) edge[above, sloped] node[black,font=\bfseries] {\tiny \texttt{}} (d);
\draw[line width=2]  (a) edge[above, sloped] node[black,font=\bfseries] {\tiny \texttt{}} (b);
\draw[line width=2]  (a) edge[above, sloped] node[black,font=\bfseries] {\tiny \texttt{}} (c);

%%%%
\draw (a) edge[line width=3, darkgreen, above, ->, bend right = 20] (b);
\draw (b) edge[line width=3, darkgreen, above, ->, bend left = 20] (c);
\draw (c) edge[line width=3, darkgreen, above, ->, bend left = 20] (d);
\draw (d) edge[line width=3, darkgreen, above, ->, bend left = 20] (g);
\draw (g) edge[line width=3, darkgreen, above, ->, bend left = 20] (i);
%%%%

\node[left = 0.1cm of a, draw, fill=green] {\footnotesize $x_1$};
\node[left = 0.1cm of b, draw, fill=green] {\footnotesize $x_2$};
\node[above = 0.1cm of c, draw, fill=green] {\footnotesize $x_3$};
\node[above = 0.1cm of g, draw, fill=green] {\footnotesize $x_4$};
\node[above = 0.1cm of i, draw, fill=green] {\footnotesize $x_5$};
\end{tikzpicture}
\end{center}
\caption{Minimal path segmentation example.}
\label{fig:min-seg1}
\end{figure}

\begin{observation*}
In a previous example we saw that $\langle \node{a}, (\node{b}, \node{c}), \node{g}, \node{i} \rangle$ is a segmentation of $p$. Our algorithm
produced instead the segmentation $\langle \node{a}, \node{b}, \node{c}, \node{g}, \node{i} \rangle$. Both these segmentations are minimal segmentations of 
$(\node{a}, \node{b}, \node{c}, \node{d}, \node{g}, \node{i})$ on the network shown in Figure \ref{fig:min-seg1}. This shows that \emph{minimal segmentations are not unique}.
\end{observation*}

An alternative way to think about our algorithm is to imagine that we start following path $p$ on the shortest path subnetwork rooted at its origin.
We do so until we either reach an edge $e$ that forces us to move out of the current shortest path subnetwork or a node with in-degree in the subnetwork
larger than $1$ (ECMP). In both cases we need to exit the current shortest path subnetwork either by adding a node segment on $e^1$ and continuing the walk
on $\sp(e^1)$ or by adding an adjacency segment over $e$ and continuing the walk over $\sp(e^2)$.

In the remainder of this section, we are going to provide a formal description of the minimum path segmentation
algorithm, give its time complexity and prove its correctness.

We start by analyzing all the situations where we need adjacency segments.

\begin{proposition}
\label{prop:adj-segs}
Let $G$ be a network and $p$ a path on $G$. Suppose that $e = (u, v) \in E(p)$ is such that either $e$ is not a shortest path
between $u$ and $v$ or there exists a shortest path from $u$ to $v$ that does not contain $e$. Then any segmentation of $p$ contains the
adjacency segment $e$. We denote this set of adjacency segments as $\adj(G, p)$.
\end{proposition}

\begin{proof}
Let $\sr{p} = \langle x_1, \ldots, x_l \rangle$ be a segmentation of $p$ that does not contain $e = (u, v)$ in its segmentation.
Then it means that there exists $i$ such that $e$ is on the shortest path between $x^2_i$ and $x^1_{i + 1}$. By hypothesis,
there exists a shortest path that does not contain $e$ from $u$ to $v$. Replacing $e$ by that path yields another shortest path
between $x^2_i$ and $x^1_{i + 1}$ showing that $\sr{p}$ is not deterministic and thus not a segmentation.
\end{proof}

\begin{algorithm}[t]
\small
\caption{$\textsf{min-seg}\left( G, p = (e_1, \ldots e_l) \right)$}
\begin{algorithmic}[1]
\STATE $r \gets p.\textsf{firstNode}()$
\STATE $\sr{p} \gets \langle \rangle$
\FOR{$e = (u, v) \in (e_1, \ldots e_l)$} \label{line:min-seg-for}
  \IF{$e \notin E(\sp(G, r))$ \textbf{or} $\delta^-(\sp(G, r), v) > 1$} \label{line:min-seg-if1}
    \IF{$e \notin E(\sp(G, u))$ \textbf{or} $\delta^-(\sp(G, u), v) > 1$}
      \STATE $\sr{p}.\textsf{append}(e)$ \label{line:min-seg-add2}
      \STATE $r \gets v$ \label{line:min-seg-r2}
    \ELSE
      \STATE $\sr{p}.\textsf{append}(u)$ \label{line:min-seg-add1}
      \STATE $r \gets u$ \label{line:min-seg-r1}
    \ENDIF
  \ENDIF
\ENDFOR
\IF{$|\sr{p}| = 0$ \textbf{or} $\sr{p}.\textsf{origin}() \neq p.\textsf{origin}()$}
  \STATE $\sr{p}.\textsf{addFirst}(p.\textsf{origin}())$  \label{line:min-seg-add3}
\ENDIF
\IF{$\sr{p}.\textsf{destination}() \neq p.\textsf{destination}()$}
  \STATE $\sr{p}.\textsf{addFirst}(p.\textsf{destination}())$ \label{line:min-seg-add4}
\ENDIF
\RETURN $\sr{p}$
\end{algorithmic}
\label{algo:minseg}
\end{algorithm}

The next two lemmas give conditions upon which we can \emph{remove} a node segment and an adjacency from a segmentation of a path $p$
such that the result it still a segmentation of $p$.

\begin{lemma}
\label{lemma:min-seg-remove-node}
Let $G$ be a network and $p = (e_1, \ldots, e_l)$ a path on $G$. Let $\sr{p} = \langle x_1, \ldots x_l \rangle$ be a segmentation of $p$ such that
$x_i \in V(G)$ for some $i$. Then $\langle x_1, \ldots, x_{i - 1}, x_{i + 1}, \ldots, x_l \rangle$ is a segmentation of $p$
if and only if there is a unique shortest path between $x^2_{i - 1}$ and $x^1_{i + 1}$ that passes by node $x_i$.
\end{lemma}

\begin{proof}
$(\Rightarrow)$ Assume that $\sr{q} = \langle x_1, \ldots, x_{i - 1}, x_{i + 1}, \ldots, x_l \rangle$ is also a segmentation
of $p$. Then $\sr{q}$ is deterministic so there is a unique shortest path between $x^2_{i - 1}$ and $x^1_{i + 1}$. To see that it passes by
node $x_i$ we observe that by definition $\seq(\sr{p}) = \seq(\sr{q})$ so
\begin{align*}
\sp(x^2_{i - 1}, x^1_i) \oplus x_i \oplus \sp(x^2_{i}, x^1_{i + 1}) = \sp(x^2_{i - 1}, x^1_{i + 1}).
\end{align*}

$(\Leftarrow)$
Since there is a unique shortest path from $x^2_{i - 1}$ and $x^1_{i + 1}$ and this path passes by node $x^1_i = x^2_i$
we have that 
\begin{align*}
\sp(x^2_{i - 1}, x^1_{i + 1}) & = \sp(x^2_{i - 1}, x^1_i) \oplus \sp(x^1_i, x^1_{i + 1}) \just{unique sp passes by $x^1_i$} \\
 & = \sp(x^2_{i - 1}, x^1_i) \oplus \sp(x^2_i, x^1_{i + 1}) \just{$x^2_i = x^1_i$} \\
 & = \sp(x^2_{i - 1}, x^1_i) \oplus x_i \oplus \sp(x^2_i, x^1_{i + 1}) \just{def of $\oplus$ and $x^1_i = x_i = x^2_i$}
\end{align*}
Therefore,         
\begin{align*}
p & = \seq(\sr{p}) \\
  & = x_1 \oplus \ldots \oplus x_{i - 1} \oplus \sp(x^2_{i - 1}, x^1_i) \oplus x_i \oplus \sp(x^2_i, x^1_{i + 1}) \oplus x_{i + 1} \oplus \ldots x_l \\
  & = x_1 \oplus \ldots \oplus x_{i - 1} \oplus \sp(x^2_{i - 1}, x^1_{i + 1}) \oplus x_{i + 1}  \oplus \ldots x_l \\ 
  & = \seq \langle x_1, \ldots, x_{i - 1}, x_{i + 1}, \ldots, x_l \rangle
\end{align*}
meaning that $\langle x_1, \ldots, x_{i - 1}, x_{i + 1}, \ldots, x_l \rangle$ is a segmentation of $p$.
\end{proof}

We have the following analogous result for adjacency segments.

\begin{lemma}
\label{lemma:min-seg-remove-edge}
Let $G$ be a network and $p = (e_1, \ldots, e_l)$ a path on $G$. Let $\sr{p} = \langle x_1, \ldots x_l \rangle$ be a segmentation of $p$ such that
$x_i \in E(G)$ for some $i$. Then $\langle x_1, \ldots, x_{i - 1}, x_{i + 1}, \ldots, x_l \rangle$ is a segmentation of $p$
if and only if there is a unique shortest path between $x^2_{i - 1}$ and $x^1_{i + 1}$ that contains edge $x_i$.
\end{lemma}

\begin{proof}
The proof is analogous to the proof of Lemma \ref{lemma:min-seg-remove-node}. 
\end{proof}

The next two results characterize the first and last segments of any segment of a path $p$. They are quite intuitive and say that 
a segmentation of a path $p = (e_1, \ldots, e_k)$ must either start with a node segment $e^1_1$ or an adjacency segment over $e_1$ and end 
with either a node segment $e^2_k$ or an adjacency segment over $e_k$.

\begin{lemma}
\label{lemma:firstseg}
Given a graph $G$ and a path $p = (e_1, \ldots, e_k)$ on $G$. Let $\sr{p} = \langle x_1, \ldots, x_l \rangle$ be a segmentation of $p$. Then
either $x_1 = e^1_1$ or $x_1 = e_1$
\end{lemma}

\begin{proof}
Since $\sr{p}$ is a segmentation of $p$ we have that $\seq(\sr{p}) = p$. Therefore, by definition of $\seq(\sr{p})$, either $x_1 = e_1$ or $x_1$ is a node segment and
$e_1$ is the first edge of $\sp(x^2_1, x^1_2)$. In the latter case we must have $x^1_1 = e^1_1$.
\end{proof}

\begin{lemma}
\label{lemma:lastseg}
Given a graph $G$ and a path $p = (e_1, \ldots, e_k)$ on $G$. Let $\sr{p} = \langle x_1, \ldots, x_l \rangle$ be a segmentation of $p$. Then
either $x_l = e^2_k$ or $x_l = e_k$
\end{lemma}

\begin{proof}
Since $\sr{p}$ is a segmentation of $p$ we have that $\seq(\sr{p}) = p$. Therefore, by definition of $\seq(\sr{p})$, either $x_l = e_k$ or $x_l$ is a node segment and
$e_k$ is the last edge of $\sp(x^2_{l - 1}, x^2_l)$. In the latter case we must have $x^2_l = e^2_k$.
\end{proof}

% \begin{proof}
% $(\Rightarrow)$ Assume that $\sr{q} = \langle x_1, \ldots, x_{i - 1}, x_{i + 1}, \ldots, x_l \rangle$ is also a segmentation
% of $p$. Then $\sr{q}$ is deterministic so there is a unique shortest path between $x^2_{i - 1}$ and $x^1_{i + 1}$. To see that it contains edge
% $x_i$ we observe that by definition $\seq(\sr{p}) = \seq(\sr{q})$ so
% \begin{align*}
% \sp(x^2_{i - 1}, x^1_i) \oplus x_i \oplus \sp(x^2_{i}, x^1_{i + 1}) = \sp(x^2_{i - 1}, x^1_{i + 1}).
% \end{align*}
% 
% $(\Leftarrow)$
% Since there is a unique shortest path from $x^2_{i - 1}$ and $x^1_{i + 1}$ and this path contains edge $x_i$
% we have that 
% \begin{align*}
% \sp(x^2_{i - 1}, x^1_{i + 1}) = \sp(x^2_{i - 1}, x^1_i) \oplus x_i \oplus \sp(x^2_i, x^1_{i + 1})
% \end{align*}
% Therefore,         
% \begin{align*}
% p & = \seq(\sr{p}) \\
%   & = x_1 \oplus \ldots \oplus x_{i - 1} \oplus \sp(x^2_{i - 1}, x^1_i) \oplus x_i \oplus \sp(x^2_i, x^1_{i + 1}) \oplus x_{i + 1} \oplus \ldots x_l \\
%   & = x_1 \oplus \ldots \oplus x_{i - 1} \oplus \sp(x^2_{i - 1}, x^1_{i + 1}) \oplus x_{i + 1}  \oplus \ldots x_l \\ 
%   & = \seq(\langle x_1, \ldots, x_{i - 1}, x_{i + 1}, \ldots, x_l \rangle)
% \end{align*}
% meaning that $\langle x_1, \ldots, x_{i - 1}, x_{i + 1}, \ldots, x_l \rangle$ is a segmentation of $p$.
% \end{proof}

To prove that our minimum segmentation algorithm is correct, we will show that any minimal segmentation
can be transformed in a series of steps to match the output of our algorithm. The next lemma will be important
in one of those transformation steps. Contrary to Lemma \ref{lemma:min-seg-remove-node}, it tells us a condition under wich we can
\emph{add} a segment to an existing segmentation of a path $p$ while preserving the fact that it is a segmentation of $p$.

\begin{lemma}
\label{lemma:min-seg-add}
Let $G$ be a network and $p$ a path on $G$. Let $\sr{p} = \langle x_1, \ldots, x_l \rangle$ be a segmentation of $p$.
For any $i = 1, \ldots, l - 1$, if $v \in V(\sp(x^2_i, x^1_{i+1}))$ then $\langle x_1, \ldots, x_i, v, x_{i + 1}, \ldots, x_l \rangle$
is a segmentation of $p$.
\end{lemma}

\begin{proof}
Since $\sr{p}$ is deterministic and $v \in V(\sp(x^2_i, x^1_{i+1}))$ it holds that
$$
\sp(x^2_i, x^1_{i + 1}) = \sp(x^2_i, v) \oplus v \oplus \sp(v, x^1_{i + 1}).
$$
Therefore,
\begin{align*}
p & = \seq(\sr{p}) \\
  & = x_1 \oplus \ldots \oplus x_i \oplus \sp(x^2_i, x^1_{i + 1}) \oplus x_{i + 1} \oplus \ldots x_l \\
  & = x_1 \oplus \ldots \oplus x_i \oplus \sp(x^2_i, v) \oplus v \oplus \sp(v, x^1_{i + 1}) \oplus x_{i + 1} \oplus \ldots x_l \\
  & = \seq(\langle x_1, \ldots, x_i, v, x_{i + 1}, \ldots, x_l \rangle).
\end{align*}
\end{proof}

% \begin{lemma}
% Let $G$ be a network and $p$ a path on $G$. Let $\sr{p} = \langle x_1, \ldots, x_l \rangle$ be a segmentation of $p$.
% For all $i = 1, \ldots, l - 1$ such that $x^2_i \neq x^1_{i + 1}$, the sr-path $\langle x_1 \ldots, x_i \rangle$ is not a segmentation of $p$.
% \end{lemma}
% 
% \begin{proof}
% \begin{align*}
% \seq(\langle x_1, \ldots, x_i \rangle) & = x_1 \oplus \ldots \oplus x_i \oplus \sp(x^2_i, x^1_{i + 1}) \oplus x_{i} \\
% \end{align*}
% \end{proof}

\begin{proposition}
\label{prop:min-seg-feasible}
Given a graph $G$ and a path $p$ on $G$, Algorithm \ref{algo:minseg} outputs a segmentation $\sr{p}$ of $p$.
\end{proposition}

\begin{proof}
Let $p = (e_1, \ldots, e_l)$, $(v_1, v_2, \ldots, v_{l + 1})$ be the sequences of nodes visited by $p$ and
$\sr{p} = \langle x_1, \ldots, x_r \rangle$. We start by proving by induction on $j \geq 2$ that $\langle x_1, \ldots, x_j \rangle$ is
deterministic and $\seq \langle x_1, \ldots, x_j \rangle$ is a prefix of $p$.

\emph{Base case:} $j = 2$.

By construction we have that either $x_1 = v_1$ or $x_1 = e_1$. This is so because the only three lines where this element 
could have been inserted into $\sr{p}$ are lines \ref{line:min-seg-add1}, \ref{line:min-seg-add2} when $e = e_1$ or line \ref{line:min-seg-add3} at the end.
If it was on one of the lines \ref{line:min-seg-add1} or \ref{line:min-seg-add3} then $x_1 = v_1$. Otherwise $x_1 = e_1$.

\emph{Case 1:} $x_1 = v_1$. Then the algorithm proceeds by iterating over the elements $e \in E(p)$ and only add the next element when we reach some edge $e_i = (u, v) = (v_i, v_{i + 1})$ 
such that there exist multiple shortest paths between $r = v_1$ and $v_{i + 1}$ or $e_i$ does not belong to any shortest path starting at $r$.
At this point we add $x_2$ which is either
equal to $v_i$ or $e_i$. In either case, since condition on line \ref{line:min-seg-if1} ensures the uniqueness of the shortest path between
$x^2_1 = v_1$ and $x^1_2 = v_i$ it holds that $\langle x_1, x_2 \rangle$ is deterministic. To see that it corresponds to a prefix of $p$, we consider the two cases for $x_2$.
If $x_2 = v_i$ then
$$
x_1 \oplus \sp(x^2_1, x^1_2) \oplus x_2 = v_1 \oplus \sp(v, v_i) \oplus v_i = \sp(v_1, v_i) = e_1 \oplus \ldots \oplus e_{i - 1}
$$ 
and if $x_2 = e_i$ then
$$
x_1 \oplus \sp(x^2_1, x^1_2) \oplus x_2 = \sp(v_1, v_i) \oplus e_i = e_1 \oplus \ldots \oplus e_{i - 1} \oplus e_i.
$$

\emph{Case 2:} $x_1 = e_1$. In this case everything remains the same except that $r = v_2 = x^2_1$. Therefore, if $x_2 = v_i$ then
$$
x_1 \oplus \sp(x^2_1, x^1_2) \oplus x_2 = e_1 \oplus \sp(v_2, v_i) = e_1 \oplus \ldots \oplus e_{i - 1}
$$ 
and if $x_2 = e_i$ then
$$
x_1 \oplus \sp(x^2_1, x^1_2) \oplus x_2 = e_1 \oplus \sp(v_2, v_i) \oplus x_2 = e_1 \oplus e_2 \oplus \ldots \oplus e_{i - 1} \oplus e_i.
$$

In any case, $\langle x_1, x_2 \rangle$ is deterministic and $\seq \langle x_1, x_2 \rangle$ is a prefix of $p$.  
This completes the proof of the base case. \\

\emph{Induction step:} Suppose that for some $j \geq 2$ the sr-path $\langle x_1, \ldots, x_j \rangle$ is deterministic and
a prefix of $p$. Using analogous arguments as above, we can show that if $e_i$ is the edge on the for loop in line \ref{line:min-seg-for} when
$x_j$ is added and $e_k$ ($j < k$) the one when $x_{j + 1}$ is added then $\langle x_j, x_{j + 1} \rangle$ is deterministic with
\[ \seq \langle x_j, x_{j + 1} \rangle                                   =
  \begin{cases}
    e_i \oplus \ldots \oplus e_{k - 1}     & \quad \text{if } x_{j + 1} = v_k \\
    e_i \oplus \ldots \oplus e_{k}  & \quad \text{if } x_{j + 1} = e_k
  \end{cases}
\]
Therefore $\langle x_1, \ldots, x_j, x_{j + 1} \rangle$ is deterministic and 
\[ \seq \langle x_1, \ldots, x_j, x_{j + 1} \rangle  =
  \begin{cases}
    e_1 \oplus \ldots \oplus e_{k - 1} & \quad \text{if } x_{j + 1} = v_k \\
    e_1 \oplus \ldots \oplus e_{k}  & \quad \text{if } x_{j + 1} = e_k
  \end{cases}
\]
is a prefix of $p$. This concludes the proof that for $j \geq 2$, $\langle x_1, \ldots, x_j \rangle$ is deterministic and $\seq \langle x_1, \ldots, x_j \rangle$ is a prefix of $p$. \\

It remains to show that at the end of the algorithm $\sr{p}$ covers the whole path. Let $x$ be the last element of $\sr{p}$
after the for loop ends. If $x^2 = v_{l + 1}$ then there is 
nothing to prove. Otherwise at line \ref{line:min-seg-add4} we add one node segment $x_l = v_{l + 1}$. With the same argument as above,
we know that $\sp(x^2, x_l)$ is unique and covers the remaining edges of $p$ since the condition on line \ref{line:min-seg-if1} was always false for the remaining edges.
\end{proof}

\begin{definition}
Let $G$ be a network and $p$ a path on $G$ visiting nodes $(v_1, \ldots, v_l)$. Given two nodes $v, u \in V(p)$ we write
$v \leq_p u$ if and only if $v = v_i$ and $u = v_j$ with $i \leq j$.
\end{definition}

\begin{lemma}
\label{lemma:min-seg-middle}
Let $G$ be a network and $p$ a path on $G$. Let $\sr{p} = \langle x_1, \ldots, x_l \rangle$ be the segmentation of $p$ produced by Algorithm \ref{algo:minseg}.
Let $\sr{q} = \langle y_1, \ldots, y_r \rangle$ be any other segmentation of $p$. Then for any $i \in \{ 1, \ldots, l - 1 \}$,
there exists $j \in \{ 1, \ldots, r \}$ such that $x^2_i \leq_p y^2_j \leq_p x^1_{i + 1}$.
\end{lemma}

\begin{proof}
Let $i \in \{ 1, \ldots, l - 1 \}$.
Suppose that no such $j$ exists. Then, since $\sr{q}$ is a segmentation of $p$, by Lemmas \ref{lemma:firstseg} and \ref{lemma:lastseg}
there exists $j$ such that $y^2_j <_p x^2_i$ and $x^1_{i + 1} <_p y^1_{j + 1}$.
But then, since $\sr{q}$ is deterministic, there exists a unique shortest path between
$y^2_j$ and $y^1_{j + 1}$. This path will contain the shortest path between $x^2_i$ and $x^1_{i + 1}$ plus, at least, the next edge of $p$ with source
at node $x^1_{i + 1}$. This contradicts the fact that $x_{i + 1}$ is a segment of $\sr{p}$ since condition \ref{line:min-seg-if1} would prevent
it to be added to $\sr{p}$. Figure \ref{fig:min-seg-middle} illustrates this proof.
\end{proof}

\begin{figure}
\begin{center}
\begin{tikzpicture}
\def\dx{1.2}
\node[scale=0.1] (v1) at (0*\dx, 0)  {\router{$v_1$}{router}};
\node[scale=0.1] (v2) at (1*\dx, 0)  {\router{$v_2$}{router}};
\node[scale=0.1] (v3) at (2*\dx, 0)  {\router{$v_3$}{router}};
\node[scale=0.1] (v4) at (3*\dx, 0)  {\router{$v_4$}{router}};
\node[scale=0.1] (v5) at (4*\dx, 0)  {\router{$v_5$}{router}};
\node[scale=0.1] (v6) at (5*\dx, 0)  {\router{$v_6$}{router}};
\node[scale=0.1] (v7) at (6*\dx, 0)  {\router{$v_7$}{router}};
\node[scale=0.1] (v8) at (7*\dx, 0)  {\router{$v_8$}{router}};
\node[scale=0.1] (v9) at (8*\dx, 0)  {\router{$v_9$}{router}};
\node[scale=0.1] (v10) at (9*\dx, 0) {\router{$v_{10}$}{router}};

\draw[line width=2]  (v1) edge[below, sloped, ->] node {\tiny $e_1$} (v2);
\draw[line width=2]  (v2) edge[below, sloped, ->] node {\tiny $e_2$} (v3);
\draw[line width=2]  (v3) edge[below, sloped, ->] node {\tiny $e_3$} (v4);
\draw[line width=2]  (v4) edge[below, sloped, ->] node {\tiny $e_4$} (v5);
\draw[line width=2]  (v5) edge[below, sloped, ->] node {\tiny $e_5$} (v6);
\draw[line width=2]  (v6) edge[below, sloped, ->, orange] node (e6) {\tiny $e_6$} (v7);
\draw[line width=2]  (v7) edge[below, sloped, ->] node {\tiny $e_7$} (v8);
\draw[line width=2]  (v8) edge[below, sloped, ->] node {\tiny $e_8$} (v9);
\draw[line width=2]  (v9) edge[below, sloped, ->] node {\tiny $e_9$} (v10);

\def\dy{0.2}
\node[above =\dy cm of v3] (xi)   {\footnotesize $x^2_i$};
\node[above =\dy cm of v6] (xi+1) {\footnotesize $x^1_{i + 1}$};
\node[above =\dy cm of v2] (yj)   {\footnotesize $y^2_{j}$};
\node[above =\dy cm of v8] (yj+1) {\footnotesize $y^1_{j + 1}$};

\draw[|-|] (xi.south) -- (xi+1.south);
\draw (yj.north) edge[|-|, above] node {\footnotesize unique shortest path} (yj+1.north);
\draw[line width=0.1cm, red!50!black] (5*\dx, 1.05) -- (6*\dx, 1.05);
\node[red!50!black] at (5.5*\dx, 1.05) {\Large \textsf{X}};

\node [draw, dashed, fill=lightgray, below= 0.3cm of e6] (text) {%
 \begin{varwidth}{3cm}
   \footnotesize $x_{i + 1} \in \sr{p}$ with $x^1_{i + 1} = v_6$ means that there is no unique shortest path starting at $x^2_i$ and using $e_6$
  \end{varwidth}
};

\node [draw, dashed, fill=lightgray] at (5.5*\dx, 2.3) (text2) {%
 \begin{varwidth}{3cm}
   \footnotesize {\color{red!50!black}\textbf{contradiction}} cannot be in a unique shortest path starting at or before $x^2_i$
  \end{varwidth}
};

\draw[dashed] (5.5*\dx, 1.05) -- (text2);

\draw[dashed] (text) -- (e6);
\end{tikzpicture}
\end{center}
\caption{Illustration of what happens if there is no element $y_j$ such that $x^2_i \leq_p y^2_j \leq_p x^1_{i + 1}$ on the proof of Lemma \ref{lemma:min-seg-middle}.}
\label{fig:min-seg-middle}
\end{figure}

\begin{lemma}
\label{lemma:ajd-min}
Let $G$ be a network and $p$ a path on $G$. There exists a minimal segmentation of $p$ such that all its adjacency segments
belong to $\adj(G, p)$.
\end{lemma}

\begin{proof}
Let $G$ be a network and $p$ a path on $G$. Let $\sr{p} = \langle x_1, \ldots, x_l \rangle$ be a minimal segmentation of $p$. Assume that 
$x_i$ is an adjacency not in $\adj(G, p)$. Then $x_i = e \in E(p)$ such that
$e$ is the unique shortest path between $(e^1, e^2)$. Therefore if we let $\sr{q} = \langle x_1, \ldots, x_{i - 1}, e^1, e^2, x_{i + 1}, \ldots, x_l \rangle$
we have
\begin{align*}
\seq(\sr{q}) & = x_1 \oplus \ldots x_{i - 1} \oplus e^1 \oplus \sp(e^1, e^2) \oplus e^2 \oplus x_{i + 1} \oplus \ldots \oplus x_l \\
             & = x_1 \oplus \ldots x_{i - 1} \oplus \sp(e^1, e^2) \oplus x_{i + 1} \oplus \ldots \oplus x_l \\
             & = x_1 \oplus \ldots x_{i - 1} \oplus e \oplus x_{i + 1} \oplus \ldots \oplus x_l \\
             & = \seq(\sr{p}) = p. \\
\end{align*}
We conclude that $\sr{q}$ is a segmentation of $p$. Moreover, since $\cost(\sr{q}) = (\cost(\sr{p}) - 2) + 2 = \cost(\sr{p})$ 
$\sr{q}$ is a minimal segmentation of $p$. By repeating this process we can obtain a minimal segmentation of $p$
such that all of its adjacency segments are in $\adj(G, p)$.
\end{proof}

\begin{lemma}
\label{lemma:ajd-algo}
Let $G$ be a network and $p$ a path on $G$. 
The set of adjacency segments in a segmentation computed by Algorithm \ref{algo:minseg} on input $p$ is $\adj(G, p)$.
\end{lemma}

\begin{proof}
Let $\sr{p} = \langle x_1, \ldots, x_l \rangle$ be segmentation produced by Algorithm \ref{algo:minseg} on input $p$.  Suppose that
$x_i = e = (u, v) \in E(p)$ is an adjacency segment. The only place where adjacency segments are added is on line \ref{line:min-seg-add2}.
This happens if and only if $e \notin E(\sp(G, u))$ or $\delta^-(\sp(G, u), v) > 1$. If $e \notin E(\sp(G, u))$ then it means that $e$ is not the shortest
path between $u$ and $v$. Otherwise, if  $\delta^-(\sp(G, u), v) > 1$ it means that $e$ is not the unique shortest path between its endpoints.
Therefore $e \in \adj(G, p)$.
\end{proof}

\begin{theorem}
Given a network $G$ and a path $p$ on $G$, Algorithm \ref{algo:minseg} outputs a minimal segmentation $\sr{p}$ of $p$.
\end{theorem}

\begin{proof}
Let $\sr{p} = \langle x_1, \ldots, x_l \rangle$ be the segmentation produced by Algorithm \ref{algo:minseg} and 
$\sr{q} = \langle y_1, \ldots, y_r \rangle$ a minimal segmentation. By Lemmas \ref{lemma:ajd-min} and \ref{lemma:ajd-algo} we can
assume that $\sr{p}$ and $\sr{q}$ have the same adjacency segments. In particular, by Lemma \ref{lemma:firstseg} we have that $x_1 = y_1$.
Since $\sr{p}$ and $\sr{q}$ have the same adjacency segments, if $y_2$ is an adjacency segment then so is $x_2$ and $x_2 = y_2$. Otherwise
both are not segments. Lemma \ref{lemma:min-seg-middle} says that, $x^2_1 \leq_p y^2_2 \leq_p x^1_2$. Let $i$ be the largest index such that
$x^2_1 \leq_p y^2_i \leq_p x^1_2$. Then $x_2$ belongs to the unique
shortest path between $y^2_i$ and $y^1_{i + 1}$. Hence, by Lemma \ref{lemma:min-seg-add},
$\langle y_1, \ldots, y_i, x_2, y_{i + 1}, \ldots y_r \rangle$ is a segmentation of $p$. Finally, by applying Lemma \ref{lemma:min-seg-remove-node} 
on $y_2, \ldots, y_i$ we conclude that $\langle y_1, x_2, y_{i + 1}, \ldots y_r \rangle$ is also a segmentation of $p$.

With this argument we were able to replace the first segment of $\sr{q}$ not matching $\sr{p}$ with the corresponding segment of $\sr{p}$.
By repeating this process for each node segment of $\sr{q}$ we are able to transform $\sr{q}$ into $\sr{p}$ while preserving the segment cost.
Therefore $\sr{p}$ is also a minimal segmentation of $p$.
\end{proof}

In this section we provided an algorithm that is able to compute minimal segmentations efficiently. The time needed to compute the
segmentation of a path depends on how we implement Algorithm \ref{algo:minseg}. If we do not precompute anything beforehand, we
will need to perform at most $V(G)$ shortest path computations. If this is done is $O(|E(G)| \cdot \log(|V(G)|)$ with Dijkstra's 
algorithm we get a complexity of $O(|V(G)| \cdot |E(G)| \cdot \log(|V(G)|)$. However, if we precompute the all pairs shortest IGP
shortest path matrix with Floyd Warshal's algorithm in $O(|V(G)|^3)$ we can then segment any path in linear time $O(|G|)$. Both these implementations
are provided in our library.


% Let $\sr{p} = \langle x_1, \ldots, x_l \rangle$ be the segmentation produced by Algorithm \ref{algo:minseg} and 
% $\sr{q} = \langle y_1, \ldots, y_r \rangle$ a minimal segmentation. Proposition \ref{prop:min-seg-feasible} tell us that
% $\sr{p}$ is a segmentation of $p$. By construction and since $\sr{q}$ start by observing that $\sr{p}$ cannot be a proper prefix
% of $\sr{q}$ nor can $\sr{q}$ be a proper prefix of $\sr{p}$. 
% 
% 
% 
% Let $i$ be the smallest index such that $x_i \neq y_i$.
% \todo{do the case where this index does not exists so that one is a subsequence of the other, not trivial but not hard}
% Let the sequence of nodes visited by $p$ be $(v_1, \ldots, v_t)$.  \\
% 
% Assume that $x^1_i = v_j$ and that $y^1_i = v_k$ with $k > j$. 
% 
% We are going to show that this is not possible.
% Let $r = x^2_{i - 1} = y^2_{i - 1}$. Because of line \ref{line:min-seg-if1}, for $x_i$ to be an element of $\sr{p}$, it means $e = (v_j, v_{j + 1})$ is such that
% $e \notin E(\sp(G, r)$ or $\delta^-(\sp(G, r), v_{j + 1})$. 
% 
% \emph{Case 1:} $e \notin E(\sp(G, r)$. In this case $\sr{q}$ cannot be a segmentation of $p$ since it fails to cover edge $e$.
% This is so because since $k > j$, it would have to do so on $\sp(y^2_{i - 1}, y^1_i) = \sp(r, v_k)$ which can only happen if $e \in E(\sp(G), r)$.
% 
% \emph{Case 2:} $\delta^-(\sp(G, r), v_{j + 1})$. In this case there is ECMP between $r$ and $v_{j + 1}$ so, since $k > j$ there is also ECMP between 
% $r = y^1_{i - 1}$ and $v_k = y^2_{i}$. Thus $\sr{q}$ is not deterministic. 
% 
% Therefore, $y^1_i = v_k$ with $k \leq j$. \\
% 
% Now we are going to prove that if we replace $y_i$ by $x_i$ in $\sr{q}$, $\sr{q}$ remains a segmentation of $p$. Since $\sr{q}$ is a segmentation
% of $p$, we have that $\sp(y^2_i, y^1_{i + 1})$ is a unique shortest path. 

