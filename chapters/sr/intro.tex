\section*{Introduction}

As mentioned in the introduction, segment routing \cite{Filsfils_SR:2015} is a new forwarding architecture that is being developed within the Internet Engineering Task Force and network operators.
Segment Routing changes the way packets are forwarded
inside a network to enable network operators to have better
control on the path followed by the packets. Traffic can be forced to follow a series of detours
which can either correspond to passing by a specific router or network link.

This chapter is dedicated to formalizing segment routing. We provide, to the best of our knowledge, the first formalization that comprises
both node and adjacency segments. We define minimal segmentations and provide an efficient algorithm for computing 
them. We also provide reachability concepts which allow to analyze the capability of a given network topology
to support segment routing as well as giving lower bound on the minimum number of segments needed reach every single link
in the network. These concepts will be fundamental later on when we propose an algorithm for computing cycle covers 
of a network.

% Figure \ref{fig:srformal_sr1} illustrates an example of SR. In this example the ingress node is node 
% $\node{a}$ and there are two segments in the SR stack: an adjacent segment representing link $(\node{d}, \node{e})$
% and a node segment representing node $\node{i}$. We assume in the figure that all IGP weights are
% equal to $1$. The ingress node will look at the segment on the top of the stack and find link
% $(\node{d}, \node{e})$. It will then forward the packet to trough origin of the link, node $\node{d}$, through the shortest
% path $(\node{a}, \node{c}, \node{d})$. Then node $\node{d}$ will receive it an forward it to $\node{e}$ one the link $(\node{d}, \node{e})$.
% Node $\node{e}$ will then examine the segment stack see that the next segment is node $\node{i}$. It then
% forwards the packet to node $\node{i}$ via the shortest path $(\node{e}, \node{h}, \node{i})$.
% 
% \begin{figure}[H]
% \begin{center}
% \begin{tikzpicture}
% \def\x{0}
% \def\y{0}
% \node[scale=0.15] (a) at (0.5 + \x,  0.5 + \y) {\router{a}{green}};
% \node[scale=0.15] (b) at (0.5 + \x, -1.0 + \y) {\router{b}{lightgray}};
% \node[scale=0.15] (c) at (2.5 + \x,  0.0 + \y) {\router{c}{lightgray}};
% \node[scale=0.15] (d) at (4.5 + \x,  0.0 + \y) {\router{d}{green}};
% \node[scale=0.15] (e) at (4.0 + \x, -2.0 + \y) {\router{e}{green}};
% \node[scale=0.15] (g) at (6.0 + \x,  0.5 + \y) {\router{g}{lightgray}};
% \node[scale=0.15] (i) at (8.0 + \x,  0.0 + \y) {\router{i}{green}};
% \node[scale=0.15] (h) at (7.0 + \x, -1.5 + \y) {\router{h}{lightgray}};
% \node[scale=0.15] (f) at (4.0 + \x, -3.5 + \y) {\router{f}{lightgray}};
% \node[scale=0.15] (j) at (8.0 + \x, -2.5 + \y) {\router{j}{lightgray}};
% \draw[line width=2] (f) edge[above, sloped] node[black,font=\bfseries] {\tiny \texttt{}} (j);
% \draw[line width=2] (h) edge[above, sloped] node[black,font=\bfseries] {\tiny \texttt{}} (j);
% \draw[line width=2] (a) edge[above, sloped] node[black,font=\bfseries] {\tiny \texttt{}} (b);
% \draw[line width=2] (b) edge[above, sloped] node[black,font=\bfseries] {\tiny \texttt{}} (c);
% \draw[line width=2] (e) edge[above, sloped] node[black,font=\bfseries] {\tiny \texttt{}} (c);
% \draw[line width=2] (b) edge[above, sloped] node[black,font=\bfseries] {\tiny \texttt{}} (e);
% \draw[line width=2] (b) edge[above, sloped] node[black,font=\bfseries] {\tiny \texttt{}} (f);
% \draw[line width=2] (e) edge[above, sloped] node[black,font=\bfseries] {\tiny \texttt{}} (f);
% \draw[line width=2] (h) edge[above, sloped] node[black,font=\bfseries] {\tiny \texttt{}} (f);
% \draw[line width=2] (g) edge[above, sloped] node[black,font=\bfseries] {\tiny \texttt{}} (i);
% \draw[line width=2] (i) edge[above, sloped] node[black,font=\bfseries] {\tiny \texttt{}} (h);
% \draw[line width=2]  (d) edge[above, sloped] node[black,font=\bfseries] {\tiny \texttt{}} (g);
% \draw[line width=2]  (d) edge[above, sloped] node[black,font=\bfseries] {\tiny \texttt{}} (e);
% \draw[line width=2]  (e) edge[above, sloped] node[black,font=\bfseries] {\tiny \texttt{}} (h);
% \draw[line width=2]  (g) edge[above, sloped] node[black,font=\bfseries] {\tiny \texttt{}} (h);
% \draw[line width=2]  (c) edge[above, sloped] node[black,font=\bfseries] {\tiny \texttt{}} (d);
% \draw[line width=2]  (a) edge[above, sloped] node[black,font=\bfseries] {\tiny \texttt{}} (b);
% \draw[line width=2]  (a) edge[above, sloped] node[black,font=\bfseries] {\tiny \texttt{}} (c);
% 
% %%%%
% \draw (a) edge[line width=2, darkgreen, above, ->, bend right = 20] (c);
% \draw (c) edge[line width=2, darkgreen, above, ->, bend right = 20] (d);
% \draw (d) edge[line width=2, darkgreen, above, ->, dotted] (e);
% \draw (e) edge[line width=2, darkgreen, above, ->, bend left = 20] (h);
% \draw (h) edge[line width=2, darkgreen, above, ->, bend left = 20] (i);
% 
% 
% \def\x{-0.25}
% \def\y{1}
% \node at (\x + 2.2, \y + 0.25) {\footnotesize SR stack};
% \fill[lightgray] (\x, \y) rectangle (\x + 1.5, \y + 0.5);
% \fill[green] (\x, \y) rectangle (\x + 1, \y + 0.5);
% \draw[dotted, thick] (\x + 0.4, \y) -- (\x + 0.4, \y - 0.4);
% \draw[] (\x, \y) rectangle (\x + 1, \y + 0.5);
% \draw[] (\x, \y) rectangle (\x + 1.5, \y + 0.5);
% \draw (\x + 1, \y) -- (\x + 1, \y + 0.5);
% \node at (\x + 0.5, \y + 0.25) {\footnotesize $(d, e)$};
% \node at (\x + 1.25, \y + 0.28) {\footnotesize $i$};
% 
% \def\x{3.25}
% \def\y{-3}
% \fill[gray] (\x, \y) rectangle (\x + 1.5, \y + 0.5);
% \fill[green] (\x + 1, \y) rectangle (\x + 1.5, \y + 0.5);
% \draw[dotted, thick] (\x + 0.4, \y + 0.8) -- (\x + 0.4, \y + 0.5);
% \draw[] (\x, \y) rectangle (\x + 1, \y + 0.5);
% \draw[] (\x, \y) rectangle (\x + 1.5, \y + 0.5);
% \draw (\x + 1, \y) -- (\x + 1, \y + 0.5);
% \node at (\x + 0.5, \y + 0.25) {\footnotesize $(d, e)$};
% \node at (\x + 1.25, \y + 0.28) {\footnotesize $i$};
% 
% 
% \def\x{8 - 0.75}
% \def\y{0.5}
% \fill[gray] (\x, \y) rectangle (\x + 1.5, \y + 0.5);
% \draw[dotted, thick] (\x + 0.4, \y) -- (\x + 0.4, \y - 0.4);
% \draw[] (\x, \y) rectangle (\x + 1, \y + 0.5);
% \draw[] (\x, \y) rectangle (\x + 1.5, \y + 0.5);
% \draw (\x + 1, \y) -- (\x + 1, \y + 0.5);
% \node at (\x + 0.5, \y + 0.25) {\footnotesize $(d, e)$};
% \node at (\x + 1.25, \y + 0.28) {\footnotesize $i$};
% 
% 
% %%%%
% 
% %\def\x{-1}
% 
% %\draw[line width=2] (\x + 10, \y + -1 + 0.5) edge[] (\x + 10.5, \y + -1 + 0.5);
% %\node[anchor=west]  at (\x + 10.5, \y + -1 + 0.5) {\footnotesize network link};
% 
% %\draw[line width=2] (\x + 10, \y + -1) edge[dotted, darkgreen, ->] (\x + 10.5, \y + -1);
% %\node[anchor=west]  at (\x + 10.5, \y + -1) {\footnotesize adjacency segment};
% 
% %\draw[line width=2] (\x + 10, \y + -1 - 0.5) edge[] (\x + 10.5, \y + -1 - 0.5);
% %\node[anchor=west]  at (\x + 10.5, \y + -1 - 0.5) {\footnotesize s link};
% 
% 
% \end{tikzpicture}
% \end{center}
% \caption{Illustration of SR with segments $(d, e)$ and $i$ and ingress node $a$. Dashed arrows represent adjacency segments
% and the others represent the shortest path edges between consecutive segments.}
% \label{fig:srformal_sr1}
% \end{figure}
% 
% This chapter is organized as follows. We start in Section \ref{section:sr-formal} by providing a formalization
% of segment routing. To the best of our knowledge, this is the first work that studies segment routing in
% 
% \todo{finish chapter organization}
% 
% \todo{we need to explain that when the graph is undirected we don't draw both arrows}
